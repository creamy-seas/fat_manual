% -*- TeX-master: "../fat_manual.tex" -*-

\section{RHUL West Bond Bonding Machine}
 
\begin{enumerate}
\item \cmd{Turn on the machine by flicking the switch near \mi{Did you book your session?}}
\item \cmd{Turn on  pump under machine} -  \red{it will not make  a sound as it  only needs to
    pressurize the tank occasionally. It is a passive element}\ec;
\item \cmd{\red{Turn the valve horizontal}\ec} to connect the pump tank;
\item \cmd{Secure sample};
\item  Typical  settings for  Aluminum  or  Gold.   \red{\textbf{Aluminum  is easier  to  bond
      with}}\ec:
  \begin{itemize}
  \item Buffer 1 or 3;
  \item Power 311;
  \item Time 33ms.
  \end{itemize}
\item Touch = one peek = begin bond,\hspace{1cm} Touch again = double peek = terminate bond;
\item \red{\textbf{Move only up or down!}};
\item If  the wire  pops out  or breaks  \red{do not pull  the wire  out from  the tube  - its
    impossible to put back in}.  \cmd{Flick \mi{Open}\ra put the wire from the back though the
    very end of the tip} (where the hole is) \ra \cmd{Flick to \mi{feed}};
\item Once done \cmd{pump off \ra valve to vertical closed position \ra power off.}
\end{enumerate}

If not bonding try to:
\begin{itemize}
\item Rewire;
\item Remove needle with Alan key \red{making sure not to drop it!}
\item Clean the needle with acetone, IPA and ultrasound \ra dry to clean the very tip;
\item Replace needle, so top bit sticks out a little bit.
\end{itemize}

\red{Bong ground planes a much as possible, and 6 bonds for transmission lines as seen below}
 
 \begin{center}
   \includegraphics[height=3cm]{bonding}
 \end{center}
 
 \newpage
 
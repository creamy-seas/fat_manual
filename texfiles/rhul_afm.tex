\section{RHUL Atomic Force Microscope}
 Positioned at RHUL it can be operated in:
 \begin{itemize}
 	\item \textbf{Contact mode} where tip scans along surface and the height the tip is moved up and down in monitored. \red{Uses single tip cantilever};
 	\item \textbf{Tapping mode} where the cantilever vibrates up and down and using a lock in technique we measure the topographical effects that map out the surface;
 	\item \textbf{Interleave mode} where there are two scans, first one for the topography, second that uses the initial scan data to work out the magnetisation.
 \end{itemize}

 \begin{enumerate}
 	\item \cmd{Turn on boxes, bottom to top on the back \ira then the TV.}
 	\item \cmd{Load up program \quote{V613b21}}
 	\item \cmd{Place tip holder onto sample holder. \red{The tip must be facing up}};
 	\item \cmd{Unscrew laser on the RHS (the pin actually goes \red{in} to release the clamp!)\ira put in the tip holder};
 	\item \cmd{\quote{V613B} \ira \quote{open workspace} \ira \quote{4thtapping} \ira click on the yellow microscope} to turn on the optical microscope;
 	\item \cmd{Choose \quote{scan triple} to monitor 3 pieces of data} e.g. height;
 	\item Align laser (while it is unscrewed from the base)
 	\begin{enumerate}
 		\item \cmd{Turn dials on top of laser until the laser light falls on the tip (you can see shadow of tip)} to position the laser light on top of the cantilever. You will evidently see a circle split in half The infromation on the screen shows the RMS voltage \ira \red{maximise it};
 		\item \cmd{Turn dials on the side and monitor the red dot on the screen cross-hairs} to position the laser light detector. \red{Should read 2-6V}. There should be a red laser line in the window on the laser module;
 		\item Additionally on the screen, the laser light should be at the center of the cantilever;
 	\end{enumerate}
 	\item \cmd{\quote{Navigate} tool and move up and down arrows to move the sample to position by moving the stage};
% 	\item \cmd{\quote{Tren focus on the cantilever (which should be around a similar height,)}}
 	\item \cmd{Click \quote{Locate tips} and click arrows} to adjust focus on cantilever;
 	 	\item \cmd{Dials on microscope (the far left hand dials facing towards you) to move optical image on the screen so that the two crosses fall along the base of the cantilever tip triangle} \red{Try to put the optical microscope so that the tip is on the blue dot};
 	\item \cmd{\quote{Focus surface}} to focus on the sample surface by combination of movement and z up-down;
 	\item \cmd{Lower illumination to check that laser shines on the middle of cantilever};
 	\item \cmd{\quote{Tune} \ira set the start and end freq of the natural cantilever frequency \ira autotune to find it} Typical value around 85kHz;
 	\item \cmd{\red{Engage} to turn scan on \ira set \quote{amplitude setpoint}  \red{to half the one set!}($ \approx 1.2V $) and \quote{scan rate} to adjust};
 	\item \textbf{The red and blue trace lines should be identical}Can play with
 	\begin{itemize}
 		\item \red{\cmd{Change units 10th option from bottom from \quote{volts} to \quote{metric}}}
 		\item \quote{trace and retrace} to choose what data each channel collects;
 		\item \quote{Line direction} 
 		\item \quote{Samples per line $ \approx $ lines = 512};
 		\item \quote{Scan size} $ \approx 2\,\mu $m;
 		\item \quote{Integral/proprtional gain} adjust by $ \pm 0.3 $ to get maximal alignment of the trace and retrace;
 	\end{itemize} 
 	\item To move tip \cmd{\quote{offset} \ira drag white square to required region \ira \quote{execute}} Note that the movement is not good, its \red{better to enter the parameters manually in the boxed. + moves the sample up/right relative to the cantilever. - moves the sample down/left relative to the cantilever (down/left on the screen).} Up/down is limited to 2$ \mu m $, left/right is limtied to $ 1\mu m $;
    \item If stucture is missed (offset can be $ \pm1\mu m $ in the X direction and $ \pm10\mu  m$ in the Y direction)
 	\item \textbf{Capture} to take the image;
 	\item To turn off \cmd{exit program \ira shut down blocks top to bottom \ira shut down computer. \red{Restart in the opposite way}};
 	\item \quote{Gwyddion} to analyse images. The images from the AFM are saved in the \quote{C:\\capture} folder;
 \end{enumerate}
 
 \newpage
% -*- TeX-master: "../fat_manual.tex" -*-
\section{RHUL The chook-chook}
\begin{framed}\noindent
  Valve 2 - its the equlibrator for  V1 - you need the same pressure on
  either side of V1, and you use V2 to do this.  Then, the can open V1.
\end{framed}
\begin{framed}\noindent
  The cooling system is composed of two parts
  \begin{itemize}
  \item The pulse  tube that fires in  helium and expands it  - it uses
    the big compressor in the shed.
  \item The circulating cycle of He3 that is pumped out of He4. This is
    the big  contour on the  right of the  big screen. P3  monitors the
    pressure  before the  capillary constriction  that this  mixture is
    funneled through.
  \end{itemize}
\end{framed}

\subsection{Pumping}
System needs to be pumped to avoid formation of ice when it is cooled.
\begin{enumerate}
\item  Can load  the  two blueforce  programs on  the  desktop or  mash
  buttons on the big panel;
\item \cmd{First stop venting by closing valves 14, 16, 19};
\item \cmd{Close the cases, with the  stitch facing the far wall of the
    lab};
\item \cmd{Check the pressure is > 40Bar on the cylinder in the corner}
  and  \cmd{80psi on  the output}  since this  is used  to operate  the
  valves;
\item \cmd{Refill the  trap with nitrogen so its full}.   The trap will
  absorb gases that are not Helium  - they liquify?  Eventually we need
  to replenish it;
\item \cmd{Turn  on the  Scroll Pump  2 \ira V21,  V16, V14}  for rough
  pumping. \red{\cmd{Wait until \quote{P6} reads 1mbar}}.
\item \textbf{Pumping  contaminated mixture  in condenser  line (around
    \cmd{V4,  V5}) and  still line  (\cmd{V3}):} \cmd{\red{ensure  that
      scroll is only  pumping through V21 and that there  is no mixture
      in  the DR  loop} \ira  open v17,  v3, V2  (to equalise  pressure
    across the  big valve) to  pump the still side  and V5 to  pump the
    condenser line side};
\item \textbf{Start  pumping the  trap} \cmd{\ira \red{close  V14, V16}
    \ira open V17 and V7} (\cmd{V21} will still be open) \ira \cmd{wait
    15-30 minutes}.

  When we  circulate the helium  in the cycle,  we pass it  through the
  Nitrogen trap (the tank).  Nitrogen,  and oxygen will be condensed in
  this trap, so only volatile things  like hygdroen and out helium will
  pass  through.  This  means that  at  the capillaries  we don't  have
  oxygen/nitrogen plugs forming.

  At room temperature we must pump the trap to remove the gases that we
  adsorbed onto the carbon mesh inside the trap.

  \begin{framed}\noindent
    \red{Position the  trap in the the  can! DO NOT TAKE  OUT EVEN WHEN
      TOPPING UP!}
  \end{framed}

\item \cmd{\red{Close  {V17} and {V7}.   Pump the can again  by opening
      V16 \ira V14. Wait until \quote{P6} = 1mbar}};
\item \textbf{Activate the turbo pump:}  \red{Close V14, V16, V21} \ira
  \cmd{Open {V23} \ira turn on Turbo2};
\item \cmd{Go outside and press  start/stop on the left-most bottom box
    \ira  Open the  dump valve  by turning  the black  knob (which  you
    cannot see) clockwise};
\item \cmd{Open V22, {V16, {V14}}};
\item \cmd{Turn on P1 on via the MaxiGauge board};
\item \cmd{Dip the trap into the nitrogen tank};
\item \cmd{After getting $ 3\times 10^{-3}\,$mBar \ira programming \ira
    load    \ira   installation    package$\backslash$control   scripts
    $\backslash$with                     extra                    turbo
    $\backslash$LD\_auto\_cooldown\_wlN2\_extra\_turbo\_V1\_21\_V23\_open
    \ira run};
\item \textbf{Good  parameters are 17\ideg input  water, 33\ideg output
    water, 35\ideg oil colour.}\red{If required, go to pumps in back of
    room, and select constant pressure.}
\item  \red{In the  morning the  can  should have  been pumped  already
    (V14-16-21 closed).   We need  to close the  turbo and  scroll pump
    \ira  \cmd{Go outside  and switch  turbo  off \ira  close turbo  on
      computer panel  \ira close  backing V23 \ira  close \quote{scroll
        pump}}}
\end{enumerate}

\subsection{Mixing chamber circulation}
\label{sec:mixing-chamb-circ}

So, the first part of the cooling  process is done with the pulse tube,
cooling to 5K, after which the He3  is pumped through He4 in the mixing
chamber.  \red{This is driven on  the right-hand-side of the panel, and
  it's the second turbo that causes this cirulation.}

In  order  for there  to  be  enough He3  circulation,  \red{\textbf{we
    actually need  to heat the still  stage so that the  mixing chamber
    cools quicker.}}

\begin{enumerate}
\item Thus, once still temperature (Ch5) reaches 600\,mK (and flow rate
  lowers  to  30\,mmol/s)  \ira   \cmd{set  ``Heat  Analog2/Still''  to
    10\,mW};
\item \red{\textbf{Make sure flow rate is 60\,mmol/s}};
\item  \red{\textbf{Make   sure  still  temperature   is  approximately
      850\,mK}}.
\end{enumerate}

\noindent  This will  ensure that  the mixing  chamber gets  enough He3
circulation to cool to the base temperature of 13\,mK.

\subsection{Shutting down}
\begin{enumerate}
\item \cmd{\quote{Maxigauge}  \ira \quote{sensor P1}  \ira \quote{off}}
  to shut down sensors during warmup;
\item \cmd{In the program load \quote{warmup} recipe \ira run};
\item  \cmd{\red{When \quote{MaxiGauge}  \ira  \quote{Channel 5}  reads
      800\,mBar get readt}};
\item  \cmd{Proceeding  clockiwse,  close  all of  the  values  in  the
    cirulation cycle V7\ira  V4\ira V1\ira V3\ira V10  \ira Scroll \ira
    V13};
\item  \cmd{Remove trap  from the  can.   \red{Close the  can with  the
      lid}};
\item \cmd{Turn off the BlueFors little box};
\item \cmd{Go outside and close the  dump valve by turning black handle
    (which you  can't see) clockwise  (from its perspective).}   All of
  the helium will be  trapped in the dump, so that it  can be reused in
  the next cycle;
\item  After  a   few  days,  you  can   vent.   \cmd{\quote{V19}  \ira
    \quote{V16} \ira \quote{V14}.}
\end{enumerate}

 \subsection{Pumping the trap}
 The trap, which distills all kinds of gasses that leak into the Helium
 system, needs to be pumped before the next cooldown:
 \begin{enumerate}
 \item \cmd{Make sure the trap is out of the canister};
 \item \cmd{\quote{V2 Scroll} \ira V21 \ira V17 \ira V7}. \red{Wait for
     1 hour!}
 \item \cmd{V7 \ira V17 \ira V21 \ira \quote{V2 Scroll}}.
 \end{enumerate}

 \subsection{Heating up the cell}
 If we want to  increase the temperature in the cell,  we need to apply
 the heater in the blue Force  program.  Standard: Control channel = 6,
 heater  =  current,  heater  resistance   =  120,  puase  time  =  60,
 proportional = 10, integral = 2, derivative = 0.
 \begin{enumerate}
 \item  \quote{Maxigauge}  \ira  \cmd{turn  off   1,  2,5  }  \ira  the
   temperature readings on the box should read \quote{disable};
 \item  \cmd{\quote{Heater   setup}  \ira  \quote{control   mode}  \ira
     \quote{closed}};
 \item \cmd{Set the \quote{setpoint} temperature you want};
 \item  \cmd{change  the   \quote{heater  range}  current  \red{1-10mA}
     depending  on  how fast  you  are  hating}  \red{You may  need  to
     increase  the  \quote{Heater Limit},  as  it  acts as  the  safety
     stopper (to not overshoot the current).};
 \item \red{\quote{Write settings}};
 \item \cmd{Monitor temperature and make sure not to overshoot.}
 \item  \cmd{\quote{Heater   setup}  \ira  \quote{control   mode}  \ira
     \quote{off}} \ira \quote{Setup} \ira \cmd{turn on 1, 2,5 }
 \end{enumerate}

 \subsection{Computer monitoring BlueForce}
 \label{sec:comp-monit-bluef}

 \begin{enumerate}
 \item To  connect to  the BlueForce  logs, \cmd{log  192.168.0.4} with
   password \cmd{resonator1}
 \item To  access monitor  connect \cmd{TightVNC viewer}  with password
   \cmd{qubit2}
 \end{enumerate}
 \newpage

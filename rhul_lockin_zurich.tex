\section{Lock-in Zurich}
A lock  in amplifier, adds a  AC signal to the  initial one, and then,  from the
output from the system, filters out  the components at the same frequency.  This
is a  good way to  remove noise.   The device is  controlled solely from  the ZI
Control Panel.  Any reading from the lock is has already been processed.

What happens is we mix in a signal with the input one

  \begin{equation}
    V_0 \rightarrow V_0+V_{L}e^{i\omega_{L}t}.
  \end{equation}

  \noindent This signal with the added AC  component is fed into the system. The
  current and voltage measured out the output will be of the form

  \begin{equation}
    V_{\text{out}} = \sum_{n}V_n^{\text{out}}e^{i\omega_n t+\delta_n},
  \end{equation}

  \noindent  which is  a mixture  of signals  from different  noise frequencies.
  Somewhere the signal at the lock in frequency in encoded - the actual response
  of the  system we want  to measure. The lock  in multiples $  V_{\text{out}} $
  with the original signal and integrates over a set time period

  \begin{equation}
    \begin{aligned}
      &    \frac{1}{T}\sum_n    \int_{0}^{T}V_{L}V^{\text{out}}_ne^{i\delta_n}e^{i(\omega_n    -
        \omega_{L})t}dt \\& =  \sum_n V_{L}V^{\text{out}}_ne^{i\delta_{n}}\int_{0}^{T}e^{i(\omega_n -
        \omega_{L})t}\frac{dt}{T}  \\&  = \sum_n  V_{L}V^{\text{out}}_ne^{i\delta_{n}}\delta_{L,n}
      \\& = V_{L}V^{\text{out}}_Le^{i\delta_L},
    \end{aligned}
  \end{equation}

  \noindent where we can remove $ V_L $ which is known, leaving

  \begin{equation}
    V_{L}^{\text{out}}e^{i\delta_L} \equiv V_{\text{signal}}e^{i\delta},
  \end{equation}

  \noindent the noise free signal from  the system i.e.  the amplitude and phase
  of the systems oscillations due  to the changing driving current/voltage.  All
  the oscillations at other frequencies have been averaged out.

  The lock  will present this  value as either  the XY components,  or R, $  \theta $
  components.

  \subsection{Tuning parameters on control panel}
  \begin{itemize}
  \item Signal inputs
    \begin{itemize}
    \item \texttt{Preamplifiers} {\ira None.  \textbf{Be  sure to select this in
          the options, otherwise incorrect demodulation will occur} e.g.  choose
        \quote{DIR} not \quote{PRE} in options;}
    \item \texttt{Scaling+units} {\ira Not needed;}
    \item \texttt{Range/Sensitivity}  \ira Its  the peak  to peak  voltage input
      (from the system we are measuring) that  will be boosted to 10V within the
      lock in.  E.g if  3V is selected,  it is  boosted to 10V  for the  lock in
      calculations, but then normalised back to  3V. \red{I used 5mV for dV, and
        10mV for dI in Jan 2017}.  You want to make this value the biggest value
      of the signal in the system e.g.  if signal is 100mV, put it at 200mV.
    \item  \texttt{AC   or  Diff}  \ira   what  type  of  measurements   we  are
      making. Usually AC.
    \end{itemize}
  \item Frequency
    \begin{itemize}
    \item  \texttt{Internal}   \ira  The   internal  frequency  that   is  mixed
      in. \red{19.0101Hz}
    \item \texttt{Demodulators}  \ira There  are 6  blocks in  the lock  in that
      perform  the   demodulation  (extraction   of  signal   at  the   lock  in
      frequency).  Tick the  ones that  are  going to  be used  for each  input.
      Select the time for this.  {33ms to 238ms.}  \red{Ensure that this time is
        \textbf{much smaller} than the time  of your measurement on LabView.  If
        we are stepping the  magnetic field per say every 1ns,  then the lock in
        will be averaging for far too long!}

      The   demodulators    are   mapped    out   ouput    \cmd{Aux1/2/3/4}   in
      ``Auxillary/IO''.   Normally  we  choose  Demodulator1  and  Demodulator4.
      Twist the \textbf{phase shift} in order to maximise X or Y signal (or just
      choose $R$).
    \end{itemize}
  \item Filters
    \begin{itemize}
    \item 24dB/Oct filters
    \item \texttt{Time constant for averaging} is discussed above.
    \end{itemize}
  \item Mixing component  - there are two  boxes. With each box  you control one
    add/out  module.  For  each  module  you   apply  a  DC  signal  (feed  into
    \texttt{add}), mix in the AC component and give it out at \texttt{out}.
    \begin{itemize}
    \item \texttt{Set range} for the maximum output voltage;
    \item \texttt{Peak  to peak  AC} \ira  the peak  to peak  voltage of  the AC
      signal. Make such its smaller than the original signal.
    \item \texttt{Add AC component} \ira ON
    \item \texttt{Turn output on} \ira ON
    \end{itemize}
  \item Auxillary  OI -  select which  demodulator results  you output  to which
    output gates. Select X, Y, R or $\Theta$.
  \end{itemize}

  \subsection{Cable connections}
  \begin{itemize}
  \item \cmd{Connect the  DC cable with the  bias signal (if any) to  one of the
      \quote{Add} inputs}. This is the initial signal, to which the AC component
    will be added;
  \item \cmd{Draw  a cable from \quote{Out}  port of the Lock-In  \textbf{to the
        system}};
  \item \cmd{Draw a  readout cable\textbf{from the system}  to the \quote{+/-In}
      port of  the lock in}. This  wire carries the  signal that the lock  in is
    going to demodulate. Take care - since  no pre amplifier is used usually, do
    not connect to any pre-amplifier modules;
  \item \cmd{Read  signal filtered by  the lock in  from one of  its \quote{Aux}
      cables.}
  \end{itemize}

  \subsection{Using oscilloscope to measure DC}
  Feed in the current and voltage lines  to the oscilloscope \ira choose XY mode
  in order to plot one against the other.  If the contact is open, then current,
  $ I $, and  voltage, $ V $, will be creating a  lissajoule figure, due to them
  being shifted in phase relative to one another.


  \newpage

%%% Local Variables:
%%% mode: latex
%%% TeX-master: "fat_manual"
%%% End:

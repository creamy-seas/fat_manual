\section{New Mask Aligner}\label{sec:new-mask-aligner}

\textbf{WEC} is wedge area compensation, ensuring that wafer and mask are
parallel.

In \textbf{manual view}, can select either left, right or split fields to
display on the main screen. In \textbf{camera view} this is controlled via
software.

\subsection{Dials:}

\begin{itemize}
  \item \textbf{10} on the $x, y, \theta$ dials in centred position;
  \item Big silver dials on the two objective sides move $x$;
  \item Small dials on the two object sides fine focus for each side;
  \item Global focus dial at the top of column;
  \item Left/right light dials for illumination control on the bottom left
        panel;
  \item Global microscope manipulator on the side side of the tower;
  \item \textbf{Gas dials}:
        \begin{itemize}
          \item Nitrogen purge - if resist is sensitive to oxygen can blast
                nitrogen into area. Very rare to use;
          \item Vacuum dial for when vacuum contact;
          \item The rightmost dial should be around $.150\,\text{mPa}$ for WEC
                pressure (wedge contact pressure). This allows the head to sit
                on a cushion, allowing the substrate to be positioned in
                parallel.
        \end{itemize}
\end{itemize}

\subsection{Levers:}

\begin{itemize}
  \item Contact lever \red{Never align when in contact};
  \item Separation lever
\end{itemize}

\subsection{Contact modes}
\label{sec:contact-modes}

\begin{itemize}
  \item \textbf{Soft contact}: wedge area compensation to the mask. \red{This is
        the one that we will use the most often};
  \item \textbf{Hard contact}: wedge area compensation where nitrogen is used to
        push wafer to the mask;
  \item \textbf{Vacuum contact}: very accurate for small features. No need for
        big features $> 2\,\mu\text{m}$; In addition to above, rubber ring allows
        vacuum to suck out air between mask and wafer, giving best contact.
        Accuracy $< 1\,\mu\text{m}$;
  \item \textbf{Low vacuum contact}: nitrogen is bled into the chamber to fill
        up the vacuum.
\end{itemize}

\subsection{Loading program}
\label{sec:loading-program}

\begin{enumerate}
  \item Pick current chuck for the substrate being used. This allows accurated
        WEC;
  \item The rubber ring is only needed for vaccum contact;
  \item On screen \textbf{Mask Vacum Off/On} hold for 4 seconds;
  \item \textbf{Parameters}:
        \begin{itemize}
          \item \textbf{Align and expose}: usual;
          \item \textbf{Flood}: blasting everything;
          \item \textbf{Test Exposure}: Do multiple exposures for different
                times. Special rotator that blocks off light different sectors.
        \end{itemize}
  \item \textbf{Exposure}: Its selecting various contact modes described in
        \autoref{sec:contact-modes}. Once selected there is an option for each
        one:
        \begin{itemize}
          \item Vacuum times - how long to wait before applying full vacuum.
        \end{itemize}
  \item Choose reflected or backside light;
  \item \textbf{Contact Chuck}: select vacuum if doing with vaccum ring;
  \item \textbf{Exposure time and repetitions};
  \item \textbf{Align check} checks alignment under exposure conditions without
        exposing. Good to check if there is a shift before exposing.
\end{enumerate}

\subsection{Loading}
\label{sec:loading}

\begin{itemize}
  \item Connect vacuum tube, to suck up the mask;
  \item Plug chuck in and tighten the side bolts;
  \item Turn the handlebars on the sides to change illumination of each
        aperture;
  \item Set the coarse focus with the big dial on the tower $\rightarrow$ do small focus
        with the small dials on each column;
\end{itemize}

\subsection{Aligning}
\label{sec:aligning}

\begin{enumerate}
  \item Click WEC settings on screen $\rightarrow$ 0.5 second wait before lifting the
        head;
  \item \texttt{Close the contact lever} - the one on the side of the machine,
        which lifts the sample up.
  \item \texttt{Turn the central dial} to {push wafer into the mask} $\rightarrow$
        \texttt{open the contact lever} once this pressure is set;
  \item Now go into WEC mode $\rightarrow$ \texttt{move the separation dial} down;
  \item Move the big side levers to align both sides and the $\theta$ dial
  \item Can move the small dial on sides of microscope objectives to shift
        camera to scan views;
  \item Go back into contact mode, \texttt{move the separation dial up} $\rightarrow$
        \texttt{close the contact lever}.
\end{enumerate}

\subsection{Expose}
\label{sec:expose}

The different lamps have different power. This is fixed, so to change dose need
to change the power.

\begin{itemize}
  \item $365\,\text{nm}$ has $24\,\text{mW/cm}^{2}$;
  \item $405\,\text{nm}$ has $58\,\text{mW/cm}^{2}$;
  \item $435\,\text{nm}$ has $16.5\,\text{mW/cm}^{2}$.
\end{itemize}

\texttt{Press expose} and lamp moves out and exposes. In the old aligner it was
mercury lamp, which was a combination of all 3. But here its more precise for
certain resists. Use buttons below yellow lamp to select what kind of light to
shine. $10-20\,\text{second}$ exposure time.

\subsection{Infra-red}
\label{sec:infra-red}

\begin{enumerate}
  \item Position ``basket'' on the head instead of the usual adaptor;
  \item Put in the filters in the objectives from the side;
  \item Take the illumination cable. It has a left and a right side. Slide them
        in under the mask on the rails;
  \item Put the other ends into \textbf{IF-L} and \textbf{IF-R};
  \item Make sure microscope is down - now we are looking through the chuck on
        the mask.
  \item Close lever, move the separation dial down.
\end{enumerate}

\subsection{Turn off}
\label{sec:turn}

\begin{enumerate}
  \item Turn off lights;
  \item Move the \textbf{separation dial} up $\rightarrow$ open the \textbf{contact
    lever};
  \item Reset the $x, y, \theta$ dials back to \textbf{10}.
  \item Hold turn off for 2 seconds;
  \item Turn off the pump;
\end{enumerate}

% -*- TeX-master: "../fat_manual.tex" -*-

\newpage\section{Dipping Procedure}
\label{sec:dipping-procedure}

\begin{enumerate}
\item \cmd{Connect orange line from side of dewar to pipe on wall};
\item \texttt{Open  the valve on the  pipe and the first  valve on the
    pumps  at the  back  of the  room}  - this  will  ensure that  any
  evaporated helium is caught back to Harpal's balloon;
\item \texttt{Take reading on pump};
\item \texttt{Place orange plum on exhaust of the probe bar} - when it
  starts inflating, there is helium flowing;
\item \texttt{Open top and place probe. Close it};
\item \texttt{Slowly lower the rod} and then do measurements;
\item \cmd{After measurements can take rod  out} - no need to be slow,
  as before you were heating the helium up;
\item \texttt{Take reading on the wall of the pump};
\item \texttt{\red{Lower thin rod with glove} into the hole at the top
    of the dewar};
\item \texttt{Push to lowest point and make a mark};
\item \texttt{Move up and when the  top of the glove changes vibration
    frequency} - \red{that  is the level of helium,  so the difference
    is the height of helium in dewar};
\item \cmd{Measure the length and read off dewar to get capacity};
\end{enumerate}

\newpage

\section{RHUL Old Scanning Electron Microscope\label{sec:sem}}
\begin{framed}\noindent
  SEM is  used to  make images,  or to trace  out the  desired pattern
  loaded from nanomaker. In this section,  we describe how to load and
  prepare the SEM for tasks.

  \begin{enumerate}
  \item If  vacuum is  bad \cmd{\quote{Sample}  \ira \quote{Specimen
        chamber}  \ira \quote{stage  drawout} \ira  \quote{evac}} to
    pump the machine
  \item If  \quote{HT ON} is  not working, its because  nanomaker has
    control of  the machine.  \cmd{Go to nanomaker  \ira \quote{File}
      \ira  \quote{Exposure}  \ira   \quote{Hardware  control}  \ira
      \quote{ExtScanOff} \ira \quote{Beam On}}.
  \item If vaccuum system halted  \ira \cmd{quit program \ira turn SEM
      off and then on with key \ira} usual pumping begins;
  \item \begin{itemize}
    \item \textbf{Aperture  1} for  high detailed images.  \red{Do not
        exceed 32 spot size!}
    \item \textbf{Aperture 2} has no restriction on current
    \end{itemize}
  \item \begin{itemize}
    \item \textbf{Scan 1} for focusing
    \item \textbf{Scan 2} for general movement
    \item \textbf{Scan 3} for high quality image. \red{Click freeze to
        freeze the scan}
    \item \textbf{Scan 4} for high quality image that also freezes
    \end{itemize}
  \item RP1  - Rotary pump  first pumps  the diffusion pump,  then the
    chamber, then the diffusion pump works on the chamber
  \end{enumerate}
\end{framed}

\subsection*{Change filament}

\begin{itemize}
\item    \quote{Gun}   \ira    \quote{Filament}   \ira    \cmd{press
    \quote{Vent}} to vent gun chamber;
\item  \cmd{Lift cap  \ira pull  cone out  \ira unscrew  \red{EXTERNAL
      SCREWS}  with alan  key, release  the  flat screw  and take  out
    filament};
\item  \cmd{Clean with  stick polish  and  cotton pad  \ira wash  with
    acetone} so that there is no black;
\item \cmd{Take  new filament, put ring  on it \ira dip  into the cone
    \red{making sure that small indent  on filament is on the opposite
      end to the slot in the cone}};
\item \cmd{Tighten external  screws and flat screw \ira  use screws in
    the ``caves of the cone" to adjust filament position};
\item \cmd{Slot into SEM \ira press \quote{evac}}.
\item When the  SEM is working, \cmd{click \quote{gun}  and slide the
    four  sliders to  get the  maximum brightness}  to compensate  for
  offsets;
\end{itemize}

  \subsection*{Beam blanking}
  For \quote{Beam  ON/OFF} on Nanomaker  we use a blanker.  The small
  module responsible for it is a box made by Deben.
  \begin{itemize}
  \item Make sure that it reads \quote{Beam blanking ON 200V};
  \item \cmd{Change voltage to 200V by \quote{Plate voltage arrows}};
  \item If beam blanking is inverted in Nanomaker (beam doesn't go off
    when  pressing  \quote{beamOff})   \ira  \cmd{Press  \quote{Beam
        Status - Blank} button};
  \end{itemize}

  \subsection{Loading}
  \begin{enumerate}
  \item The loading chamber should  be vented already \ira Evac button
    on screen should be grey.
  \item \red{Put on a healthy scratch  on the resist for good focusing
      if resist thickness is greater than 100nm}
  \item Attach the sample onto the doughnut shaped container. Screw on
    well. Top of sample, must be near  the end of the doughnut that is
    \red{symmetrical}. Should look like two frog legs.  Open door \ira
    Slide  onto the  rails \red{if  it doesn't  want to  go in,  try a
      different orientation.} Close the door.
  \item \texttt{Sample \ira Evac  \ira Airlock chamber}, will evacuate
    the chamber  the sample was loaded  in to a low  pressure. It will
    beep when done and button stays green. Takes 30sec.
  \item \red{Turn  the open handle, to  open the door from  the sample
      stage to the SEM.}
  \item Raise rod  down. \red{Do no touch the  rod!!!!!!} Move forward
    {and twist}  to grab hold of  the sample. Push the  sample all the
    way  in {it  becomes hard  towards the  end} and  \red{remember to
      unlock!!!}. Draw the rod out.
  \item \red{CLOSE THE DOOR BETWEEN THE SEM AND THE SAMPLE CHAMBER!!}
  \item Wait 10 minutes for the SEM to pump to low pressure again.
  \item \texttt{Initial position  \ira Go}, to move the  sample to the
    operational area.
  \end{enumerate}

  \subsection{Calibrating}
  The following steps calibrate the SEM for operation.

  \begin{enumerate}
  \item  First of  all set  the current  and voltage  to the  required
    values.  This  is  done  at  the bottom  of  the  program  on  the
    screen. $I\approx10pA$  and $V\approx30keV$. The  Working Distance
    (WD)$\approx20mm$. For  the 2D flakes,  we were at  24pA. \red{The
      lower  the   current,  the  better  the   precision.}  Push  the
    \texttt{MDN} and \texttt{AVG} buttons  on current meter to average
    readings.
  \item  \cmd{Change working  distance to  20$  \mu $m  for general  and
      10$ \mu $m for fine structures like meanders};
  \item \red{\texttt{Turn the SEM on by clicking the HT ON button}.}
  \item Choose the  correct aperture, by pulling lever on  the side of
    the   SEM    towards   you    and   twisting   to    the   desired
    selection.   Aperture  1   is   the  smaller   -   has  the   best
    precision. \red{Do not exceed 32  spot size}. Aperture 2 is larger
    and  can sustain  spot  sizes  greater than  40  and  is used  for
    imaging.
  \item \red{\texttt{Ensure that  X/Y is chosen on  the controller for
        the movement.}}
  \item Go to  the Coulomb dot (black  dot), and zoom in  on it. Check
    that the current is the value that you selected.
  \item  \texttt{Press ACB  button},  which  automatically adjust  the
    brightness and contrast.
  \item  \texttt{Press  R  button  on   the  keyboard  and  align  the
      structure}. If  needed, \texttt{click on the  sample location on
      the SEM screen, and manually  enter the rotation in degrees that
      is required.}
  \item Zoom to $\approx$x200000 onto  a nice edgy structure like Rais
    dots
  \item \textbf{Wobbling}  - \texttt{Tool->OL Wobbler}, will  make the
    image oscillate - essentially if the aperture is at an angle, then
    going in  and out of  focus will have  the effect of  shifting the
    image. Turn the dials on the rod that sticks out the SEM to reduce
    the wobbling as much as possible. Turn the wobbler off.
  \item \textbf{Astigmatism and  focus} - \texttt{Press scan  1, to go
      to the near  field. Hold down the focus button,  and move up and
      down to focus.  If the focus is different in  the horizontal and
      vertical directions, do the same for the X and Y astigmatism.}
  \item \textbf{Gun alignment} - align the gun by \texttt{changing the
      X Y and Z}, until the current in the coulomb dot, is maximised.
  \end{enumerate}

  \subsection{Prepare for use with nanomaker}
  Now that  the SEM  is calibrated  (should be  done after  each large
  movement), we need to prepare it to communicate with nanomaker.

  \begin{enumerate}
  \item Focus on the scratch or near the sample at x100000;
  \item Go  to the  required magnification.  E.g. x4000.  For graphene
    flakes  it was  x350.  \red{\texttt{Inst Mag  on  the keyboard  to
        remember this value}}. This is useful  if you need a very high
    magnification during exposure, but you cant use it during focusing
    as it would expose the whole resist.
  \item  Move  the  the  required  structure.  \texttt{INST  MAG}  and
    \red{quickly}    \texttt{Double   click}    to   center    without
    overexposing.
  \item   \red{On    Nanomaker   in    Sec.\ref{sec:nanomaker}   click
      \texttt{BEAM OFF\ira  INST MAG}  (if required)  \texttt{\ira EXT
        SCAN ON}}
  \end{enumerate}

  \subsection{Ending session}
  \begin{enumerate}
  \item Zoom  out and  \red{\texttt{HT off} to  turn off  the electron
      gun}.
  \item \texttt{Sample \ira initial position}  to move the sample back
    for extraction.
  \item \red{Wait 5 minutes} to allow the tungsten cathode to cool. If
    you  begin venting  before  this  time, TuO  strings  form on  the
    filament,
  \item  Evacuate  the  loading chamber  \texttt{Sample  \ira  Airlock
      chamber  \ira Evac}.  It will  beep  when ready  and show  green
    button.
  \item  Open the  shutter, put  the rod  in \red{the  hold position},
    extract. Twist to unlock and place rod.
  \item \red{Close the shutter once finished.}
  \item Can take out as soon as required.
  \end{enumerate}


  \subsection{Taking image}
  \begin{enumerate}
  \item \quote{Scan 4} \ira \quote{File} \ira \quote{Save as};
  \item Take data from the computer underneath the desk.
  \end{enumerate}
  \newpage

  \section{Nanomaker\label{sec:nanomaker}}
  \subsection{Communication with SEM}
  To pass on control between the SEM and the nanomaker program:

  \begin{enumerate}
  \item  \texttt{Click  the  exposure  button,  or  go  to  File  \ira
      Exposure}.
  \item \texttt{Select  Hardware control to  open up a small  block of
      buttons}
  \item To  go from  SEM to nanomaker:  \texttt{Beam Off  \ira ExtScan
      On}.
  \item To  go from  nanomaker to SEM:  \texttt{ExtScan Off  \ira Beam
      On}.
  \item  Remember,  that  if the  scan  need  to  be  done at  a  high
    magnification,  \texttt{to  flick  the  INST  Mag  button}  before
    proceeding.
  \end{enumerate}

 \subsection{Marker files}
 Marker  files are  used to  align the  field, so  that the  design is
 exposed in the  correct area. For example, changing  the current will
 distort the image taken by the SEM.

 However,  if  one definitely  knows  the  position of  some  markers,
 e.g. crosses at $200\mu m$ separation, then one can
 \begin{itemize}
 \item Define that Marker A is at position $(200,50)\mu m$ etc.
 \item After  scanning the image with  the SEM, click on  the image to
   define where the center of Marker A is.
 \item The  program will stretch, move,  twist the image, so  that the
   positions   you  clicked   on  the   screen  get   the  coordinates
   $(200,50)\mu m$ etc.
 \end{itemize}

 \red{\LARGE  Always load  the parameter  file when  creating markers,
   opening markers etc.  This sets the correct field  size (the yellow
   border). \texttt{Options \ira load PAR.}}

 Now how to create marker files
 \begin{enumerate}
 \item \red{It  is crucial  that you know  what magnification  you are
     working on. Otherwise the position of the markers will be for the
     incorrect working field.}
 \item \texttt{File  \ira new Video  control file} to create  a marker
   file.
 \item \texttt{Marker \ira Define marker} to access the markers in the
   field. Here you set
   \begin{enumerate}
   \item The  size of  the field  that is  going to  be used.  Too big
     causes too  much exposure.  Too small,  and you  wont be  able to
     locate the markers.
   \item The coordinate of the field area.
   \end{enumerate}
   By default there is a zero marker  for the centre of the field. Use
   it to set the size of the field.
 \item \texttt{Add Marker} and define its position (usually determined
   from the  file were the  marker was designed)  and the size  of the
   image  that  is   taken  around  that  position,   to  help  during
   alignment. Create as many markers as needed.
 \item \red{Save  as a .mrk file.  Ensure that both the  file type and
     the ending .mkr are used.}
 \end{enumerate}

 To use  the marker files,  load them in  nanomaker, and make  sure we
 have control on the SEM that is positioned close to the markers.

 \begin{enumerate}
 \item  \red{\texttt{Video \ira  set align  parameters \ira  reset} to
     reset any previous  adjustments. Do not do this, if  you want the
     field orientation to remain the same.}
 \item \texttt{Video \ira Get all Video},  takes a scan of the regions
   where we defined the markers.
 \item \cmd{For fields  were markers are very close  to exposure field
     \red{for initial alignment take video on one marker to not expose
       others} i.e.  press \quote{v} instead of  \quote{a} and apply
     shift} \ira \cmd{then do the alignment with all the markers};
 \item  \cmd{If you  dont want  to use  certain markers  for aligning,
     don't set the  's' center point on them} this  may be required if
   many markers keep not aligning;
 \item For  each marker \texttt{Mark  \ira select center},  and select
   where  the  marker  is  on the  image.  \red{\texttt{Video\ira  set
       alignment  parameters  \ira  reset   zoom  for  large  patterns
       \quote{x=1,  y=1} to  prevent resizing  pattern BUT  FOR SMALL
       PATTERNS DONT DO THIS!!!!\ira calculate} to apply the changes.}
 \item \texttt{Video \ira Get all  Video}, to see how the recalculated
   field aligns  with the defined  markers. Keep repeating.  With each
   iteration,  twisting  etc  will  make sure  that  the  markers  you
   defined,   and  the   markers  you   select  on   the  screen   are
   aligned.  \red{Assumes   that  we   are  always  working   a  fixed
     magnitude.} We can no expose.
 \end{enumerate}

 \subsection{Taking an image}
 \red{\Huge Always use aperture 2 for currents above 32pA!!!!}
 \begin{itemize}
 \item \cmd{Click \quote{SRT}} to rotate the image;
 \item \texttt{File  \ira new  Video control file}  to create  a blank
   file.
 \item  \cmd{\quote{Mark}   \ira  \quote{Define  Mark}   and  select
     460x320$ \mu m $} to fill up whole sector with image;
 \item \cmd{Select resolution of image using \quote{Size [pix]}};
 \item  \texttt{Video  \ira Get  All  Video}  takes  an image  of  the
   specified size.
 \item Save as a \red{.tif and ensure that the file type is also TIF.}
 \end{itemize}

 \subsection{Pattern}
 \begin{itemize}
 \item  \cmd{From autocad,  save the  image in  format \quote{R12/LT2
       DXF}, by  using \quote{saveAs} and  clicking on the  drop down
     arrow next to the file name;}
 \item \texttt{File  \ira New  database} to create  a blank  sheet for
   working on.
 \item \red{\texttt{Options \ira  Load PAR \ira 350.par}}  to load the
   parameter  files for  resist, wafer,  magnification etc.  \red{MUST
     DO.  Choose parameter  file in  correspondence with  the required
     magnification 350 for large field, 800 for small field.}
 \item Add a background  (for example if you want to  design on top of
   an  image that  was already  taken) \texttt{View  \ira redraw  \ira
     background}
 \item  \texttt{View  \ira Draw  Layers}  in  order to  shift  between
   different layers on the image. Convenient  to have all the files in
   one place.  To create  a new layer,  draw object  \texttt{edit \ira
     change attributes and set new layer.}
 \item Draw using the standard tools. \red{If overlaps occur, then the
     regions will  be given  a higher  dose. See  proximity correction
     below to overcome this.} To repeat pattern \cmd{\quote{Transform
       numeric}  \ira   set  the   \quote{dose  increment}   at  each
     transformation}
 \item Once drawn, there are two options to set the \% doses:
   \begin{enumerate}
   \item  Apply  the   proximity  correction.  \texttt{Proximity  \ira
       Correction \ira Recommended} to get the menu to pop up. Here it
     depends on  the lithography that  is being made. But  for example
     for the 2D flakes:
     \begin{itemize}
     \item $\alpha=0.05\mu m$
     \item Resist = PMMA 30-40keV
     \item Substrate = Silicon
     \item  Height =  $1\mu m$  \red{OF THE  RESIST! Use  thickness vs
         rotation speed graphs to determine this}
       % \item Beam = 24pA
       % \item Sensitivity = 200
     \end{itemize}
     \red{\texttt{Apply  proximity  correction  by pressing  SET}}  in
     order  for the  new parameters  to  be used.  The regions  should
     change colour once this is done.
   \item For fine structures i.e. DC samples etc, the dose must be set
     manually.  Use  the  above   proximity  correction  procedure  to
     evaluate the  required doses - one  will find that the  doses are
     uniform (the structures  are so small, that one does  not need to
     account for the wide backscattered Gaussian)
     \begin{itemize}
     \item   \cmd{Select  structures   \ira  \quote{edit\ira   change
           attributes \ira set dose}}. The doses we need to set are:

       \begin{center}
         \begin{tabular}{|c|c|c|c|}
           \textbf{Structure size} & $ \ge 0.2\,\mu $m & $ 0.2\,\mu $m & $ \le 0.1\,\mu $m\\
           \textbf{Dose \%} & 166 & 181 & 200
         \end{tabular}
       \end{center}
     \end{itemize}
   \end{enumerate}
 \end{itemize}

 \subsection{Preparing and exposing parameters}
 \begin{enumerate}
 \item    \texttt{Options    \ira     Exposure    and    video    \ira
     \quote{Step/Times}}  to set  the exposure  parameters (\red{note
     that this is different from the proximity correction above. Above
     we just varied the \% dosage that we apply to shapes. Here we set
     the base dose (the 100\% from  which the absolute value of the \%
     doses  will be  derived)that the  electron beam  delivers}). This
   will vary  from lithography to  lithography, and the values  we use
   are the same as for the proximity correction e.g.
   \begin{itemize}
   \item $\alpha=0.05\mu m$
   \item Resist = PMMA 10-40keV
   \item Substrate = Silicon
   \item Height = $0.5\mu m$
   \item I = 24pA
   \item Dose = 90$ \mu $As/cm$ ^{2} $
   \item Sensitivity = 200
   \end{itemize}
   \red{\texttt{Click  OK  \ira  click   the  three  buttons  next  to
       numerical values} to set parameters!!!}
 \item If  the \quote{area dwell  time} is $  \ge 10\, \mu$s,  then you
   need to increase  the step size in  the step menu e.g.  from 0.1 to
   0.125$ \,\mu $m, so that the beam doesn't jump around too fast.
 \item \red{\texttt{Options \ira  Load PAR \ira 350.par}}  to load the
   parameter  files for  resist, wafer,  magnification etc.  \red{MUST
     DO.  Choose parameter  file in  correspondence with  the required
     magnification 350 for large field, 800 for small field.}
 \item Do the marker alignemnt described above
 \item  \texttt{Layer}  to select  the  layer  of  the drawing  to  be
   exposed.
 \item A 100\% dose corresponds to the following formula:
   \begin{equation}
     \text{Dose} = \frac{I\times \tau}{\text{Pixel}^{2}},
   \end{equation}
   \noindent where we set \textbf{Dose}, \textbf{I} in \quote{options
     \ira times  \ira recommended} and \textbf{pixel}  in \quote{step
     size}. The  dwell time  $ \tau  $ evaluated  from this  \red{must be
     $ \ge 10\mu $s or the beam will jump around too fast}
 \item \red{For small structures \cmd{\quote{Options}\ira choose x or
       y  direction \ira  \quote{simulate} (in  exposure window)}  to
     trace out how beam will jump. Choose X or Y direction where there
     are fewer jumps (beam jumping  is bad)}. Normally no optimistaion
   is the best!
 \item Finally click \quote{start} and don't touch the table.
 \end{enumerate}


 \subsection{Example design for flakes}
 \begin{enumerate}
 \item \red{\texttt{Options \ira Load PAR \ira 350.par}};
 \item \cmd{Load sample and take image as .tif};
 \item \cmd{Unload sample};
 \item   \cmd{Start   designing   (ensure    that   you   still   have
     \quote{350.par}) with the tif image as background};
 \item \cmd{Apply proximity correction;}
 \item \cmd{Use screen pointer to  find the coordinates of the markers
     that are in the image. \red{Write them down}};
 \item  \cmd{Create  marker   files,  and  put  the   markers  in  the
     coordinates taken above}  - this means that the  pattern you drew
   will be mapped to the correct markers;
 \item \cmd{Load and expose}.
 \end{enumerate}


 \newpage

\section{Chalmers Ellipsometer}
  \begin{enumerate}
  	\item \cmd{Switch on with plug at the bottom \ra turn on lamp with button that is covered with a plastic cover};
  	\item \cmd{Place sample on the stage and ensure that light falls on middle};
  	\item \cmd{Load program on computer \ra go to \mi{hardware} window and click \mi{move} in the menu bar \ra turn dials so that the cross moves to the centre} illuminating as much as possible. Use . and , for vertical movement;
  	\item \cmd{Go to \mi{acquire data} \ra \mi{set angle} \ra set to 70};
  	\item \cmd{\mi{Acquire data} \ra \mi{spectroscopy}} to take the data;
  \end{enumerate}
  \subsection{Modelling}
   \begin{itemize}
   	\item \cmd{Build up layer by layer by \mi{adding layer} and choosing file with the material and setting \mi{thickness}};
   	\item \cmd{For unknown material use the \mi{CAUCHY} file and set approximate refractive index in parameter slot \mi{A\_n}};
   	\item \cmd{Go to \mi{data} window and click \mi{generate data} - check its good};
   	\item \cmd{\mi{Manual run} \ra click on the thickness box in top window and scroll with middle mouse button to align the generated and measured spectra};
   	\item \cmd{Tick the \mi{thickness} and box other parameters you want to fit (e.g. energy, cauchy parameters) \ra click \mi{model}};
   \end{itemize}
   
   This extracts the best thickness fit and refractive index;
 
 \newpage
 
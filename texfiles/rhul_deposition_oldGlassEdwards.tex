\section{RHUL Oled Edwards depositer\label{sec:edwards}}
  \iframe{\red{Remember to follow the arrow on the handle which points to current pump configuration}
  
  \red{If thickness monitor is not responding click \mi{shutter} button mutliple times until it shows \mi{close}.}}
  
  \subsection*{Starting up}
   \begin{itemize}
   	\item \cmd{Turn on the Edwards Plug \ra open front door and press \mi{Green button}} (its the breaker);
   	\item \cmd{Turn on the pump with in a switch in the back corridor};
   \end{itemize}
  
  \subsection{Preparation}
  \begin{enumerate}
  	\item \cmd{Turn on diffusion pump} (yellow handle). It needs to be constantly pumped by another pump (we never turn it off. if needed, pull out its plug). Give 20 mintues to heat up;
  	\item \cmd{\mi{Air admit}} to vent chamber \ra red indicator lamp turns on \ra \cmd{wait 2 min};
  	\item \cmd{Remove the two bubble cases \ra \red{\mi{air admit} to off}} in order to not forget afterwards;
  	\item \cmd{Remove chimney \ra If required unscrew the plasma etch setup.} There are two fairly large screws on the bottom of the holding plate and two screws next to the protractor dial. \cmd{Fully unscrew them and place to one side \ra shfit the plasma etching setup (needle, protractor, base) to side of hole};
  	\item \cmd{Screw in the sample holder to a bar that is placed on top of the hole};
  	\item \textbf{For spiral} \cmd{get 10cm of the deposited material and wind it onto the spiral coil, ensureing good contact};
  	\item \textbf{For boat} \cmd{Top up material};
  	\item There are three electrodes in the machine. Connect the spiral and boar, one to common, one to one of the leftover electrodes
  	\begin{itemize}
  		\item \red{The far back electrode is ground = common}
  		\item Side one is 1;
  		\item Near-back one is 2;
  	\end{itemize}
  	\item \cmd{\red{Place separator between boat and spiral}.} It will be lifted up when the diffusion pump is activated and the stage rises up;
  	\item \cmd{\red{MAKE SURE THAT THE CRYSTAL USAGE ON THE THICKNESS MONITOR IS NO MORE THAN 80}};
  	\item \cmd{Place chimney \red{so that thickness monitor can see both sources}} \ra \cmd{\red{ensure that shutter is closed}};
  	\item \cmd{Place any wiring on the top stage};
  	\item \cmd{Position lids with \mi{exclamation sign} opposite a small circular pin};
    \end{enumerate}

  \subsection{Pumping}
  \begin{enumerate}
  	\item  \cmd{Turn lever \red{anticlockwise} to roughing} to begin pumping \ra check the \mi{pirani 10} and \mi{black mks box} on other depositor for pressure;
  	\item \cmd{Turn the \mi{do not use force} pin out}  - this relieves the pressure in the gas lines so that there is no leakage (controllers on the wall produce a very good seal);
  	\item Add 3 scoops of nitrogen to trap oil.
  	\item \cmd{Once \red{200\,mTorr} shows on the mks barometer and the \mi{pirani 10} (around 5min) \ra handle \red{clockwise 360 degrees} to hook up diffusion pump};
  	\item Turn on \mi{Penning 8};
  	\item \red{\cmd{If the \mi{oernning valve} doesn't rush over $ 10^{-4} $ then refill the nitrogen trap}};
  	\item \cmd{Keep track of \mi{Penning valve} \ra \red{wait for $8 * 10^{-6} torr $} which takes around 1 hour};
  \end{enumerate}	
  	
  	\subsection{Depositing}
  	 \begin{enumerate}
  	 	\item \cmd{On the thickness block \mi{layer} \ra use \mi{inc/dec} to choose the layer}. Make sure correct \mi{density} and \mi{z-value} show up;
  	 	\item \cmd{Turn the black lever to \mi{source 1} or \mi{source 2} (horizontal positions), depending on what we are going to be depositing};
  	 	\item \cmd{Press yellow \mi{LT} (low voltage) button \ra press \mi{green} safety button \ra start ramping the current up.} Typically it would reach 20-30\% for good deposition;
  	 	\item \cmd{\red{May need to turn handle anticlockwise to lower the stage - the separator might be blocking even dispersion!}}
  	 	\item \cmd{Open shutter and press \mi{open} on the thickness monitor. Once required thickness is reached, close the shutter manually};
  	 	\item \cmd{Ramp the current down to 0 \ra Press red \mi{reset} button, so that the green button jumps up \ra turn LT off};
  	 	\item \cmd{\red{Only after this, may you switch the voltage source}};
  	 \end{enumerate}
   
   \subsection{Final steps}
    \begin{enumerate}
    	\item \cmd{Turn handle to \mi{backing} until it won't turn no more} and wait for 5 minutes to allow stuff to cool \ra \cmd{turn off the penning monitor};
    	\item \cmd{Turn the \mi{do not use force} knob inwards, to cancel the atmpshpehric pressure;}
    	\item \cmd{Press \mi{air admit}} to vent the chamber;
    	\item \begin{itemize}
    		\item Remove sample and divider (if used);
    		\item Remove boat and spiral;
    		\item Rescrew the plasma setup \ra put on glass cylinder and top bit;
    		\item Put chimney back in;
    		\item Replace bubble and imposion guard.
    	\end{itemize}
    	\item \cmd{\red{Press \mi{air admit}}};
    	\item \cmd{Handle to \mi{roughing}, wait for 200mTorr reading on mks} \ra \red{DO NOT LEAVE LIKE THIS!!};
    	\item \cmd{Turn off diffusion pump};
    	\item \cmd{\red{Turn handle to backing position and leave!}}
    \end{enumerate}
  	
  
 \newpage
 
 
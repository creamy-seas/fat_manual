% -*- TeX-master: ``../fat_manual.tex" -*-

\section{EBL RHUL Jeol 8100x}
Password: Jeoleb% \subsection{Loading Sample}

\begin{framed}\noindent
  Lifehack - to close lines run the command \texttt{pedit}.
\end{framed}

\subsection{Loading and unloading}
\label{sec:loading-unloading}

\begin{itemize}
\item  \cmd{jbxwriter \ira  {Stage control}  \ira  {Fixed Position  = ORG}  \ira
    Move}.  The {Current Position} should be close to X$\approx$Y$\approx$100$\mu$;
\item \cmd{Turn the black and red lock open};
\item \cmd{Press  flashing green {LOAD/UNLOAD}  button and hold for  3 seconds}.
  You will hear valve opening;
\item \cmd{Grab  cassette - it should  naturally tip to safe  position}; \red{Do
    not touch corners  - they are used for height  measurements.  If dirty clean
    with IPA};
\item   \cmd{Load  in   the  chip   or   wafer,  performing   alignment  as   in
    \autoref{sec:marker_location}};
\item \cmd{Load cassette as below:}
  \begin{figure}[h]
    \centering
    \def\svgwidth{8cm}%
    \import{images_inkscape/}{cassette_load.pdf_tex}
  \end{figure}

\item \cmd{{Load, unload} \ira wait 3 seconds};
\item \cmd{Check {Manual Loader Viewer} in jbxwriter \ira Blue means cassette is
    loaded};
\item  Click \textbf{IV}  tab \ira  Once \textbf{V2}  and \textbf{V3}  both show
  \textbf{Open}, operation can begin;
\item It is  recommended to wait one  hour after loading for  the temperature to
  stabilise.   The  temperature  fluctuations  can be  checked  following  steps
  outlined  in  sec.~\ref{Temperaturefluctuations}.   During  the  exposure  the
  temperature  should not  change  more  than $0.05$  degrees  Celsius.  If  the
  exposure is short it might not be necessary to wait a full hour.
\end{itemize}

\subsection{Finding the markers before loading}
\label{sec:marker_location}

To align the chip in the JEOL,  you need to tell \cmd{jbxwriter} the location of
the \texttt{PQRS}  and first chip in  the coordinates of the  cassette.  This is
done on the microscope next to the door.

\begin{figure}[h]
  \centering
  \def\svgwidth{8cm}\import{images_inkscape/}{marker_position.pdf_tex}
  \caption{\small  To align  the  chip,  you will  need  to  store the  relative
    coordinates (relative to  the center of the  window they are in) of  P, Q, R
    and S.\label{fig:marker_position}}
\end{figure}

\begin{framed}\noindent
  \red{Measurements  are made  relative to  the center  of the  window you  have
    chosen}
\end{framed}

\begin{itemize}
\item Select \textbf{cassette name} e.g.  {2inch multi E};
\item Select  \textbf{window} e.g.  2E \ira  \red{The window should turn  red in
    the program};
\item Locate PQRS markers, taking down their positions.

  \begin{table}[htbp]
    \centering
    \caption{Coordinates, \label{tab:p-and-q}}
    \begin{tabular}{|l|c|c|}
      \hline
      \textbf{Marker} & \textbf{Design Location} & \textbf{Measured Location}\\\hline
      P & 0, 0 & -4333, 5050\\
      Q & 8000, -10000 & 3666, -4950\\
      Chip & +100 relative to P & -4333, 5200 \\\hline
    \end{tabular}
  \end{table}
\end{itemize}

\subsection{Creating the exposure program in JobMaker}\label{sec:jobmaker}

\subsubsection{Basic Settings}
\label{sec:basic-settings}

\begin{itemize}
\item Select Cassette name and choose window.
\item Press Job Property and a window ``Job 1 Property" will pop up.
\item In that window under the Calibration heading
  \begin{itemize}
  \item Select EOS Mode.
  \item Choose your Condition File.
  \item Choose your Calib. Menu which selects what pre-exposure calibration to perform. \textbf{TEST000} performs height detection. \red{Do not use this, if using windows A, B or C as height detection will fail}. Instead use \texttt{none}
  \end{itemize}
\item Under the Exposure Condition heading
  \begin{itemize}
  \item Choose Scan Step
  \item Enter Beam Current
  \item Choose OL Aperture
  \item Choose dose (the resist does not matter)
  \end{itemize}
\end{itemize}

\begin{framed}\noindent
  Never tick \textbf{Height Detection} - this can only be used with chip mark detection, where height is measured every time. If you need the height map, select the suitable calibration file instead (see above).
\end{framed}

\newpage
\subsubsection{Global Mark Detection}
\label{sec:glob-mark-detect}

\begin{framed}\noindent
  Layers that  need to  be aligned  with lower  layers will  need to  locate the
  markers from the previous layer. For calibration procedure see Sec.~\ref{sec:material-correction}.
\end{framed}

\begin{figure}[h]
  \centering \inkfig{12cm}{pq_coordinates}
  \caption{\small  1)   Convenient  markers  chosen  for   aligment;  2)  Marker
    coordinates    taken    down   using    microscope    next    to   SEM    in
    Sec.\ref{sec:marker_location};     3)     During    material     correction,
    Sec~\ref{sec:material-correction},  set \texttt{P(0,0)},  \texttt{Q(1000,0)}
    and set  offset to the microscope  coordinates of the \textbf{P}  point. The
    SEM  will then  measure  and find  a  more accurate  offset;  4) Supply  the
    \textbf{Offset}  and   \textbf{Design  Coordinates}   in  Jobmaker   in  the
    \textbf{Job Properties} menu.}
  \label{fig:pq_coordinates}
\end{figure}

\subsubsection{Chip Mark Detection}
\label{sec:chip-mark-detection}

\begin{framed}\noindent
  Patterns  where every  individual  pattern has  a chip  mark  for even  better
  alignment before.
\end{framed}

\begin{itemize}
\item Tick Chip Mark Detection
\item Choose Type
\item Choose Mode
\item Tick Height Detection
\end{itemize}

\subsubsection{Other}
\label{sec:other}

\begin{itemize}
\item Below  \textbf{Chip List}, click on  \texttt{...}.  A window pops  up \ira
  Choose your file and press \textbf{Open}.
\item Press on press \textbf{a, b, c, d, etc} to drop pattern onto grid.
\item Right click on it and select \textbf{Chip property}.
  \begin{itemize}
  \item Choose Center Position
  \item Choose ``Subfield Sorting" (Direction of exposure)
  \item Press  ``Shot Rank Table..."   next to Shot Rank  and a window  will pop
    up. Here you need to set the doses, enter Modulation in \%. Press Close.
  \item Enter your chip mark coordinates.  tem Choose shape of your chip mark.
  \item Enter your chip mark size.
  \item Press OK.
  \end{itemize}
\item To  create an array,  right click on your  pattern and select  ``Array". A
  window pops up.
  \begin{itemize}
  \item Choose your array size (Number)
  \item Choose your chip to chip distance (Pitch)
  \item You may want to tick Grouping,  especially if you are planning to create
    another array of the array (Nesting).
  \item Press Ok
  \end{itemize}
\item You may create another array if you wish.
\item Save File.
\item Compile (triangle icon) and press OK.
\end{itemize}
Return to ``jbx writer"
\begin{itemize}
\item \red{Before exposure click {read offset} to transfer alignment information
    from {jbwriter} to this design file};
\item Press ``Exposure" button.
\item Press open and select your Magazine File, the file you have just compiled.
\item Press ``Start Exposure"-button, press OK.
\item  You  can  monitor,  the  progress of  exposure:  Yellow  (writing),  Blue
  (completed)
\end{itemize}

% \subsection{GDS Conversion}
% If  file is  in  \texttt{gds}-format, you  first  need to  convert  it to  v30
% following  the  steps  below.   Note  that  each  layer  has  to  be  imported
% separately. Know your layers!
% \begin{itemize}
% \item Open ``jbxconv" on main console (DELL computer).
% \item Choose your file: File $\rightarrow$ Open.
% \item Choose \textbf{Structure} and \textbf{Layer}.
% \item Choose ``Size parameters".
% \item Command $\rightarrow$ Start Conversion $\rightarrow$ OK.
% \end{itemize}

% \begin{table}[htbp]
%   \centering
%   \begin{tabular}{|p{3cm}|p{3cm}|c|c|}
%     \hline
%     \textbf{Parameter} & \textbf{Meaning} & \textbf{Where to enter} & \textbf{Example}\\\hline\hline
%     Cassette name & The big metal hunk being used & \hyperref[sec:marker_location]{Marker Location}, \hyperref[sec:jobmaker]{jobmaker}, \hyperref[sec:material-correction]{Material Correction} & \texttt{2 inch}\\\hline

%     Window name & Window chosen  & \hyperref[sec:marker_location]{Marker Location}, \hyperref[sec:jobmaker]{jobmaker}, \hyperref[sec:material-correction]{Material Correction} & \texttt{2C}\\\hline

%     PQRS by design & Where they should be according to design & \hyperref[sec:jobmaker]{jobmaker} & (4000,5000) \\\hline
%     PQRS real & Where JEOL will locate them & \hyperref[sec:marker_location]{Marker Location}, \hyperref[sec:material-correction]{Material Correction} & (4334, 6000)\\\hline
%     Condition file & Aperture and Current used & \hyperref[sec:jobmaker]{jobmaker}, \hyperref[sec:material-correction]{Material Correction} & \verb|2nA_60um|\\\hline
%   \end{tabular}
% \end{table}


% \subsection{Checking temperature fluctuations}
% \label{Temperaturefluctuations}
% \begin{itemize}
% \item \cmd{Terminal \ira {ebanalyze} \ira {trend}};
% \item \cmd{Select start and end date time \ira Press search};
% \item \cmd{Click on file, it will turn black};
% \item Press ``check".
% \item To see  graph select ``MONTMPS" for temperature,  ``MONVACS" for vacuum,
%   ``MONACCS" for accelerating voltage.
% \end{itemize}

\subsection{Condition file}
\begin{framed}\noindent
  This sets the current and aperture for the JEOL.
\end{framed}
\begin{itemize}
\item  \cmd{jbwriter  \ira  {Condition}   \ira  {Condition  File  Loading}  \ira
    {Calibration Condition File Select}};
  % \item\
  %   \begin{framed}\noindent
  %     \cmd{Choose between}
  %     \begin{itemize}
  %     \item High Throughput:
  %       EOS mode 3, 100 keV,
  %       lens 8, from 2nA
  %       and above
  %     \item High Resolution:
  %       EOS mode 6, 100 keV,
  %       lens 5, 100-400
  %       pA.
  %     \end{itemize}
  %   \end{framed}
\item \cmd{For  each option, available Condition  files will appear in  a table.
    Choose the appropriate  one.  If the desired condition file  does not exist,
    ask Dr.  Shaikhaidarov to create it.}
\item Apply; %or do we have to press Apply?
\item Tick the ``Restore" and ``DEMAG" option $\rightarrow$ Press \textbf{check}
  to check that the condition file was created recently;
\item  Change aperture:  Open the  ``Main Console  Access"-door.  Pull  and turn
  cylinder handle  to the appropriate  aperture.  \red{Aperture must  be changed
    before clicking execute.}
\item Press  \textbf{execute} and the settings  from the condition file  will be
  applied;
  \item To use another condition file, copy it and load it.
\end{itemize}

\begin{table}[h]
  \centering
  \begin{tabular}{|c|c|c|c|}
    \hline
    & \textbf{Current} & \textbf{Aperture Size} &  \textbf{Lens}\\
    \textbf{High throughput} & 10-100nA & \iunit{300}{$\mu$m} - 8  & 4th Lens\\
    \textbf{High resolution} & 200pA & \iunit{60}{$\mu$m}  & 5th Lens (only for 200pA)
    \\\hline
  \end{tabular}
\end{table}

\subsection{Alignment after aperture is changed}
\label{Alignment}

\begin{itemize}
\item In the top  right of the console \textbf{Choose BE  as Fixed Position}. We
  will perform initial calibration on this marker.
\item Press \textbf{Move} $\rightarrow$ press \textbf{SEM} to activate SEM;
\item For faster scan tick \textbf{Rapid} and press \textbf{Apply};
\item Adjust \textbf{Brightness}  and \textbf{Contrast}.  It is  a good strategy
  to put the contrast to maximum, and adjust the Brightness;
\item Adjust magnification;
\item Initial calibration should be done with back-scattered electrons (BE);
\item For 2nA and 200pA current, click \textbf{x56}.
\item If nothing is visible on the  screen, move to larger aperture and turn the
  dials to align the aperture.
\end{itemize}

The outcome of this should be a cross like so:

\begin{figure}[h]
  \centering \inkfig{3cm}{jeol_cross_be}
  \caption{\small     The    goal     is    to     make    the     cross    stop
    pulsating\label{fig:jeol_cross_be}}
\end{figure}

\begin{itemize}
\item In the  \textbf{Alignment Tab} Press \textbf{WOBB} \ira Press  on the lens
  you are using (it will be lens 4 or 5);
\item Eliminate wobbling by turning dials on  the rod with the apertures stick -
  start with one to eliminate diagonal (the  closest dial) and then the other to
  eliminate  other  diagonal;  \red{The  image   should  pulsate  but  not  move
    laterally.}
\item Turn off wobbler by clicking \textbf{Wobb}, and wait for DEMAG to complete
  \ira SEM OFF.
\end{itemize}

\subsection{Calibration}

\subsubsection{Current calibration}
\label{sec:current-calibration}

\begin{itemize}
\item In the \textbf{Curr.   Adjust.tab}, tick \textbf{Beam Current Measurement}
  \red{Do no need to do the ticking and sections below most of the time}.
\item Press \textbf{Execute} and a current will be measured (check logs);
\item  If all  is good,  proceed to  repeating the  wobbling steps  described in
  Sec.~\ref{Alignment}.
\item Otherwise, follow these steps in order.
\end{itemize}

\textbf{Option 1: Shifting aperture}.
\begin{itemize}
\item  Press \textbf{Align}  \ira  a  circle with  crosshairs  should appear  on
  screen. This is the aperture and we want it to be centered.
\item Press  the \textbf{[2 al.   shift]} on  the grey console  (above keyboard,
  below screen).  Use \textbf{X} and \textbf{Y} knobs below to position aperture
  in the middle;
\item \textbf{Align} off.
\item Check current again and check wobbling.
\end{itemize}

\textbf{Option 2: Tilting to improve beam current}
\begin{itemize}
\item  Move  to  \textbf{FC} (Faraday  Cup)  in  the  top  right corner  of  the
  application;
\item Click \textbf{SPOT};
\item Click \textbf{2 al. tilt} \ira Toggle x/y to maximise current;
\item Check current again and check wobbling.
\end{itemize}

\textbf{Option 3: Demaging}
\begin{itemize}
\item  Press  \textbf{2/3} on  the  grey  console  \ira  Toggle focus  wheel  to
  increase/decrease current;
\item Once current is obtained press \textbf{DEMAG} \ira Press \textbf{2nd};
\item When \textbf{2nd} is off, press \textbf{3rd}.
\item When \textbf{3rd} is off, press \textbf{DEMAG} (it will turn off);
\item Use the focus knob to change current again;
\item Press \textbf{4th} (or \textbf{5th}) \ira \textbf{SPOT} off.
\end{itemize}

\subsubsection{Standard mark detection calibration}
\label{sec:stand-mark-detect}

\begin{itemize}
\item Tick AE and BE Mark Detection in the \textbf{STD.  Mark Detection}.
\item Press \textbf{Execute}
\item If  marks were  found, press  \textbf{Save} in  the \textbf{AE}  panel and
  \textbf{BE panel};
\item If not:
  \begin{itemize}
  \item Increase the Scan width to 40 $\mu$m and repeat mark detection;
  \item If marks are now found, press \textbf{Save};
  \item   Decrease  the   to  4$\mu$m   \ira  Press   \textbf{Apply}  \ira   Press
    \textbf{Execute} \ira Press \textbf{Save};
  \item  Can  also increase  \textbf{raster}  -  this  will increase  number  of
    averages
  \item In the {Wave}-tab change Gain and Offset;
  \item Press {Applies to Calibration} \ira {All select} $\rightarrow$ Apply;
  \item Stop Wave and try executing again;
  \item Can click on \textbf{Wave} tab and button - here you can change the gain
    to for better detection.
  \end{itemize}
  \begin{framed}\noindent
    From Rais  file, if  detection still  fails: Turn  on [SEM].   .  At  the BE
    conditio press [make current posit on default].  [save].  OK. AE: check mark
    condition to  find wha corner  to align to.  {(4,4)-bottom  left, (4,-4)-top
      left, -4)-top ight, (-4,4)-b ttom  right} Position corresponding co ner in
    the center of  the screen.  Press [Regi  OK. At the AE  conditio press [make
    current pos tion default].  Ok.  Execute dete  tion of both AE a d BE marks.
    .
  \end{framed}
\end{itemize}

If detection still fails, time to find markers manually:

\paragraph{Finding BE manually}
\begin{itemize}
\item Turn on \textbf{SEM};
  \item Set magnification to x50,000 as the standard cross will fir the grid perfectly.
\item Move to \textbf{BE} and use arrows to move marker to middle of the screen;
\item Press \textbf{Register} (in the fixed position submenu);
\item In the \textbf{BE Tab} update  the BE coordinates by clicking \textbf{Make
    current Position default};
\item Press \textbf{Save}.
\item Repeat for \textbf{under BE}
\end{itemize}

\paragraph{Finding AE Manually}

\begin{itemize}
\item Set magnification to x50,000 as the standard cross will fir the grid perfectly.
\item Check the mark condition, to find the corner to align it to:
  \begin{itemize}
  \item (4, 4): Bottom left;
  \item (4, -4): Top left;
  \item (4, -4): Top right;
  \item (-4, -4): Bottom right;
  \end{itemize}
\item Turn on  \textbf{SEM} and move crosshairs, so that  the relevant corner is
  in the center of the screen;
\item Press \textbf{Register} (in the fixed position submenu);
\item In the \textbf{AE Tab} update  the AE coordinates by clicking \textbf{Make
    current Position default};
\end{itemize}

\begin{framed}\noindent
  After manually finding AE and  BE \red{rerun automatic detection.  Remember to
    save both AE and BE!}
\end{framed}

\subsubsection{Focusing}
\label{sec:focusing}

\begin{framed}\noindent
  For large 100nA  current, you must focus yourself, because  the machine does a
  bad job. Or for very precise jobs such as 200pA.
\end{framed}
\begin{itemize}
\item Click \textbf{SEM} on and move to marker \textbf{SEM SP};
\item Set \textbf{x3000 magnification} and \red{do not use rapid scan};
\item Pull out the small table and adjust focus using the grey controller:
  \begin{itemize}
  \item Use 4th lens;
  \item Click \textbf{FINE} for finer movements;
  \item Turn the \textbf{LENS/DF} dial.
  \end{itemize}
\item Press the \textbf{Stig} button and remove astigmatism.
\item \textbf{Even better to do it on \cmd{SEM SP}}
\item \red{To not perform automatic focusing after this, simply skip this tab}.
\end{itemize}

\begin{framed}\noindent
  If not using 100na, automatic focusing will work.
\end{framed}
\begin{itemize}
\item Tick \textbf{Static Focus Correction} in the Focusing-tab.
\item Execute \ira Save
\end{itemize}

\subsubsection{Deflection correction}

\begin{itemize}
\item Tick all  options (Main DEF., Sub DEF., Dist.   corr.)  in the \textbf{DEF
    CORR} tab;
\item Execute \ira Save
\item You  may want to  check that the  calculations values for  the convergence
  judgment X,Y are below 4 nm (6 nm) for lens 5 (4) in the ``Log".
\end{itemize}

\subsubsection{Material Correction (general)}
With the height map, JEOL will make a matrix of heights - need to specify no points and the separation.

\begin{figure}[h]
  \centering
  \inkfig{3cm}{jeol_height_map}
  \caption{\small How Jeol creates matrix height map\label{fig:jeol_height_map}}
\end{figure}

\subsubsection{Material Correction (aligning the sample)}\label{sec:material-correction}

Begin by selecting correct material type (2 inch) and the window (2D or 2C) \textbf{in every submenu!}

\begin{framed}\noindent
  For layers after the first one,  aligning higher layers will require execution
  of this step.
  \begin{itemize}
  \item The  real coordinates of the  markers are measured outside  the SEM. The
    way this is done in described in Sec.~\ref{sec:marker_location}.
  \item   The   design    coordinates   of   the   markers    are   written   in
    Sec. \ref{sec:jobmaker}.
  \item The best way to deal with P and Q coordinates is to set \texttt{P=(0,0)}
    and set Q to whatever it's offset  is relative to it. This guide will assume
    that this is the logic taken.
  \end{itemize}
\end{framed}

\begin{figure}[h]
  \centering \inkfig{12cm}{pq_coordinates}
  \caption{\small  1)   Convenient  markers  chosen  for   aligment;  2)  Marker
    coordinates    taken    down   using    microscope    next    to   SEM    in
    Sec.\ref{sec:marker_location};     3)     During    material     correction,
    Sec~\ref{sec:material-correction},  set \texttt{P(0,0)},  \texttt{Q(1000,0)}
    and set  offset to the microscope  coordinates of the \textbf{P}  point. The
    SEM  will then  measure  and find  a  more accurate  offset;  4) Supply  the
    \textbf{Offset}  and   \textbf{Design  Coordinates}   in  Jobmaker   in  the
    \textbf{Job Properties} menu.}
\end{figure}

\begin{itemize}
\item Tick \textbf{Global Mark Detection};
\item Select the chip holder and window;
\item Enter  the \red{design} P  (which is always  set to \texttt{(0,0)})  and Q
  marker coordinates;
  \item \textbf{For Aluminum markers, use manual mode! Agree to locate the marker, but instead of clicking continue, press cancel. EBL will then NOT look for the markers itself, and asssume that they are under your coordinates.}
\item \
  \begin{framed}\noindent
    \red{\textbf{Most important part:}} Now  enter the measured coordinates (see
    \autoref{sec:marker_location})  of  the  \textbf{P}   point  in  the  offset
    box. What  you are telling the  machine, is that point  \textbf{P} will have
    coordinates \texttt{(0,0)} as long as this  offset is performed. You will be
    instructing the machine to shift so that \textbf{P} is located at the origin
    \texttt{(0, 0)}.
  \end{framed}
\item In the  \textbf{Global Mark} menu click \textbf{RG  Detect Condition} \ira
  in \textbf{Scan Type} enter width of mark;
\item Repeat for \textbf{Q Rough Scan, P Fine Scane, Q Fine scan};
\item Execute \ira move crosshairs to center of mark \ira Press \textbf{Save};
\end{itemize}

\begin{framed}\noindent
  \red{The machine  will cacluate a more  accurate offset, which can  be seen in
    the \textbf{Log}}. This is the  offset to be typed in \autoref{sec:jobmaker}
  (or clicking \textbf{Read Offset}).
\end{framed}

\begin{framed}\noindent
  For aluminum, the following settings are best:
  \begin{itemize}
  \item BE
  \item Add signal course derivative;
  \item 56 Coarse gain;
  \item Middle Gain 1900;
  \item BE Offset 1600;
  \item 1 dummy scan;
  \item 5 scans.
  \end{itemize}
\end{framed}

\begin{framed}\noindent
  This is also the best place to doing manual focusing.
  \begin{itemize}
  \item  After  marks  have  been  detected \ira  click  on  the  \textbf{target
      position}  coordinate   boxes  (This   will  activate  them)   \ira  click
    \textbf{Move}
  \item Now that you  are on the chip makers, zoom in  and perform focusing like
    in \autoref{sec:focusing};
  \end{itemize}
\end{framed}


\subsubsection{End of calibration}
\label{sec:end-calibration}

\begin{framed}\noindent
  Update \ira Save and exposure is ready to go.
\end{framed}

\subsection{For JEOL engineers and advanced users only}

In the {Focusing} tab
\begin{itemize}
\item Only tick {Static Focus Correction}
\item Choose SF/SSX/SSY;
\item Press Execute \ira Save;
\end{itemize}
In the ``DEF. Corr."-tab
\begin{itemize}
\item Tick {Main DEF}, {Sub DEF}, and {Dist. Corr.};
\item Allowable gain error should be 4nm;
\item Distortion error should be 6nm (4nm) for lens 4 (5);
\item Execute $\rightarrow$ OK $\rightarrow$ Save;
\end{itemize}

\subsubsection{Checking how marks are detected (this does not affec the calibration file)}
In the {Material Corr.}-tab \ira \cmd{use {semi-automatic} mode}.  \textbf{First
  Global Marks:}
\begin{itemize}
\item Have only {Global Mark Detection} ticked
\item In the {Global Mark} tab have only Q point ticked;
\item Enter the position of your Q mark of your specific wafer;
\item Execute $\rightarrow$ Ok;
\item If it fails, untick Rapid, increase scan width to maximum and press Apply;
\item When you found  the markers, tick Grid under the  SEM-tab and press apply,
  position marker to the centre, and press Continue a d Save;
\item In the {Log} check the offset and type it into the set field in the Global
  Mark tab
  \begin{framed}\noindent
    \red{Here  take note  of the  P, Q  and offset  values.  They  will be  used
      below.}
  \end{framed}
\end{itemize}
\textbf{Then Chip Marks:}
\begin{itemize}
\item Have only {Chip Mark Detection} ticked;
\item Choose a mode;
  \begin{itemize}
  \item 1 for one mark and offset, but no rotation;
  \item 4 for four marks and rotation, measures position and height;
  \item V1 and V4 for auto focus
  \end{itemize}
\item Execute \ira Update $\rightarrow$ Save
\end{itemize}

 \subsection{Creating pattern\label{subsec:jobMaker}}
 Claibrating the EBL is  all good.  We also need to prepare  the pattern that it
 ill expo e.   Do this n verb|  B Job Mak r|.   The main goals ar  : \b gin{enum
   rate} \i em efine th coordinates of the global markers:
 \begin{itemize}
 \item {Job property};
 \item Select Window position on the sample holder you are using (A, B, C ,D);
 \item Select the window size.  Big wafers are 3" and a 2x2 chip fits a 2";
 \item  Define the  coordinates of  the global  markers P  and Q  \red{using the
     coordinate and offset values you foun  during ca ibration.  Yes - just copy
     them.}
 \item \cmd{Tic other pr such as current, calibration file etc.}
 \end{itemize}

 \begin{itemize}
 \item Select the  .v30 pattern to expose  \ira Right click to make  an array of
   them;
 \item \cmd{Right  click on the  pattern} \ira  \cmd{Define the position  of the
     chip mark for the  first pa tern.  If you want to offset  the patte n to be
     in betwen the chi s \red{apply (-402,-46 set}};
 \item \cmd{Save and load it up for exposure.}
 \end{itemize}


 \begin{framed}\noindent
   For movement, make  sure that {cassette} in  the top right corner  is the one
   you chose.
 \end{framed}
 \newpage

%%% Local Variables:
%%% mode: latex
%%% TeX-master: "fat_manual"
%%% End:

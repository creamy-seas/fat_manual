\section{Dads wonderful 4 point measurement setup}
  The following setup is used for all kinds of current/voltage measurements in a system. The four terminal approach allows one to evaluate the voltage drop across the samples and the current going through it, bypassing the voltage drop across any wires in the system by using very large resistors.
  
  The scheme of operation is shown below. The setup is placed in a box, with inputs for the $ V_\text{Iout} $ and $ V_\text{out} $.
  
  \begin{figure}[h]
  	\centering
  	\includegraphics[height=9cm]{4terminal}
  	\caption{Essentially this is a four terminal measurement configuration, where the voltage is measured directly across the sample, and the current is measured via bias resistors, which all but nullify the internal resistances of the wires, to get a measurement going.}
  	\label{fig:4t}
  \end{figure}

  The setup is connected through a box as below. Pins are used to connect laboratory wires with desired sample wires. There important rules are:
  

  \begin{figure}[h]
  	\centering
  	\includegraphics[height=9cm]{box}
  	\caption{Dials are used to choose resistors and gains. The various wires to the sample area are connected to the labortoary wires via a mesh, that connects two given cables. e.g. for a two terminal measurement, 2V+ 2I+ 3I- 3V- 5G (the coordinates of the pins that we would place) and we would ground all remaining sample wires to a single ground. \red{VERY IMPORTANT TO GROUND V+ V- etc when changing pins and put all sample wires common.}}
  	\label{fig:box}
  \end{figure}
    \begin{itemize}
  	\item Pins at 6V+ connects V+ to the 6th wire/contact on the sample. Refer to a diagram to determine correct connection.
  	\item Whenever changing connections, ground V+ V- I+ I- G by putting in pins at 12V+ 12V- 12I+ 12I- 12G where the 12th wire to the sample is in some way grounded;
  	\item Whenever idle, to avoid potential difference build-up, ground all the sample wires as well e.g. put pins on 1G 2G 3G .. 20G, so that they are all grounded via the 12th sample wire;
  \end{itemize}
  \newpage
  
  \section{Connecting matrix}
The matrix allows one to connect cables in the laboratory to contacts on the samples. In the image below: \textbf{Horizontal} - laboratory cables; \textbf{Vertical} - contacts to sample.


\begin{figure}[h]
	\centering
	\begin{minipage}{0.4\linewidth}
		\includegraphics[width = \textwidth]{set1}
	\end{minipage}
	\begin{minipage}{0.4\linewidth}
		\includegraphics[width = \textwidth]{set2}
	\end{minipage}
	\begin{minipage}{0.4\linewidth}
		\includegraphics[width = \textwidth]{set3}
	\end{minipage}
\end{figure}
\begin{itemize}
	\item \textbf{First image shows} how to ground all the sample when not in use and how to common ground the laboratory cables via the bad contact. \red{When putting new pins in \ra put in new pins \ra take out the four common pins to separate them from the ground \ra take out ground pins.};
	\item Configuration for two terminal measurement with gate;
	\item Configuration for four terminal measurement with gate;
\end{itemize}

\subsection{Connection to Pre-Amp in the lab}
\begin{enumerate}
	\item \cmd{Connect the power cable from the \mi{blue box} +5V to the pre-amp};
	\item \cmd{To $ V_\text{in} $ connect the voltage cable from \mi{Signal Output 2 Out} on the Lock In};
	\item \cmd{$ V_\text{iout} $ to PXIe-4462 AI0};
	\item \cmd{$ V_\text{out} $ (has blue tags) to PXIe-4462 AI3};
	\item \cmd{Combine $ V^{+} $, $ I^{+} $ and $ V^{-} $, $ I^{-} $ together in pairs using a $ \pi $-connector for 2-point measurement, or keep separate for 4-point};
	\item \cmd{Open \mi{ziControl} and click \mi{on} to turn the lock in output on for the second channel (or the one we are using)};
	\item \cmd{Load \mi{IV\_va\_VR\ra IV\_BlueForce}};
\end{enumerate}


\newpage

  
 
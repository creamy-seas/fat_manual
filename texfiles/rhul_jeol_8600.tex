% -*- TeX-master: "../fat_manual.tex" -*-

\section{EBL RHUL Jeol 8100x}
Password: Jeoleb% \subsection{Loading Sample}

\begin{framed}\noindent
  Lifehack - to close lines run the command \texttt{pedit}.
\end{framed}

\begin{itemize} 
\item \cmd{jbxwriter  \ira \mi{Stage control} \ira  \mi{Fixed Position = ORG}  \ira Move}. The
  \mi{Current Position} should be close to X$\approx$Y$\approx$100$\mu$;
\item \cmd{Turn the black and red lock open};
\item \cmd{Press flashing green \mi{LOAD/UNLOAD} button and hold for 3 seconds}. You will hear
  valve opening;
\item  \cmd{Grab cassette  - it  should naturally  tip to  safe position};  \red{Do not  touch
    corners - they are used for height measurements. If dirty clean with IPA}\ec;
\item Load in the chip or wafer, performing alignment as in Sec.~\ref{sec:marker_location};
\item Load cassette as below
  \begin{figure}[h]
    \centering \def\svgwidth{6cm}\import{images_inkscape/}{cassette_load.pdf_tex}
  \end{figure}

\item \cmd{\mi{Load, unload} \ira wait 3 seconds};
\item \cmd{Check \mi{Manual Loader Viewer} in jbxwriter \ira Blue means cassette is loaded};
\item It is recommended to wait one hour  after loading for the temperature to stabilise.  The
  temperature    fluctuations    can    be     checked    following    steps    outlined    in
  sec.~\ref{Temperaturefluctuations}.  During  the exposure the temperature  should not change
  more than $0.05$ degrees Celsius. If the exposure is short it might not be necessary to wait
  a full hour.
\end{itemize}

\subsection{Finding the markers before loading}
\label{sec:marker_location}

To align the chip in the JEOL, you need to tell \cmd{jbxwriter} the location of the \texttt{PQRS} and first chip in the coordinates of the cassette. This is done on the microscope next to the door.

\begin{figure}[h]
  \centering
  \def\svgwidth{8cm}\import{images_inkscape/}{marker_position.pdf_tex}
  \caption{\small To align the chip, both you need to defined the PQRS global markers and chip markers.\label{fig:marker_position}}
\end{figure}

\begin{framed}\noindent
  \red{Measurements are made relative to the center of the window you have chosen}\ec
\end{framed}

\begin{itemize}
\item \cmd{Select \mi{cassette name} e.g. \mi{2inch multi E}};
\item  \cmd{Select  \mi{window}}  e.g.   2E  \ira  \red{The window  should  turn  red  in  the
    program}\ec;
\item \cmd{Locate PQRS and  the first chip marker} and take down  the positions \ira \red{make
    sure you know the real positions as well}\ec
  
  \begin{table}[htbp]
    \centering
    \begin{tabular}{|l|c|c|}
      \hline
      \textbf{Marker} & \textbf{Design Location} & \textbf{Measured Location}\\\hline
      P & -4000, 5000 & -4333, 5050\\
      Q & +4000, -5000 & 3666, -4950\\
      Chip & +100 relative to P & -4333, 5200 \\\hline
    \end{tabular}
  \end{table}
\end{itemize}

\subsection{GDS Conversion}
If file is in \texttt{gds}-format, you first need to convert
it  to v30  following the  steps below.  Note that  each
layer has to be imported separately. Know your layers!
\begin{itemize}
\item Open "jbxconv" on main console (DELL computer).
\item Choose your file: File $\rightarrow$ Open.
\item Choose \textbf{Structure} and \textbf{Layer}.
\item Choose "Size parameters".
\item    Command    $\rightarrow$    Start    Conversion
  $\rightarrow$ OK.
\end{itemize}

\begin{table}[htbp]
  \centering
  \begin{tabular}{|p{3cm}|p{3cm}|c|c|}
    \hline
    \textbf{Parameter} & \textbf{Meaning} & \textbf{Where to place} & \textbf{Example}\\\hline\hline
    Cassette name & The big metal hunk being used & \hyperref[sec:marker_location]{Marker Location}, \hyperref[sec:jobmaker]{jobmaker}, \hyperref[sec:material-correction]{Material Correction} & \texttt{2 inch}\\\hline

    Window name & Window chosen  & \hyperref[sec:marker_location]{Marker Location}, \hyperref[sec:jobmaker]{jobmaker}, \hyperref[sec:material-correction]{Material Correction} & \texttt{2C}\\\hline

    PQRS by design & Where they should be according to design & \hyperref[sec:jobmaker]{jobmaker} & (4000,5000) \\\hline
    PQRS real & Where JEOL will locate them & \hyperref[sec:marker_location]{Marker Location}, \hyperref[sec:material-correction]{Material Correction} & (4334, 6000)\\\hline
    Condition file & Aperture and Current used & \hyperref[sec:jobmaker]{jobmaker}, \hyperref[sec:material-correction]{Material Correction} & \verb|2nA_60um|
  \end{tabular}
\end{table}

  
  \subsection{\texttt{Jobmaker} to prepare pattern than JEOL draws}
  \label{sec:jobmaker}
  \begin{itemize}
  \item \cmd{Select \mi{cassette} and \mi{window}};
  \item \cmd{\mi{Job  property} \ira choose  \mi{condition file} \ira  choose \mi{calbiration
        menu}};
  \item \cmd{Set PQRS marks and chip mark};
  \item Save file and compile;
  \item \red{Before  exposure click \mi{read  offset} to transfer alignment  information from
      \mi{jbwriter} to this design file}\ec;
  \end{itemize}


\subsection{Checking temperature fluctuations}
\label{Temperaturefluctuations}
\begin{itemize}
\item \cmd{Terminal \ira \mi{ebanalyze} \ira \mi{trend}};
\item \cmd{Select start and end date time \ira Press search};
\item \cmd{Click on file, it will turn black};
\item Press "check".
\item To  see graph  select "MONTMPS"  for temperature, "MONVACS"  for vacuum,  "MONACCS" for
  accelerating voltage.
\end{itemize}

\subsection{Condition file}
\textbf{This sets the current and aperture for the JEOL}.
\begin{itemize}
\item \cmd{jbwriter \ira \mi{Condition} \ira \mi{Condition File Loading} \ira \mi{Calibration
      Condition File Select}};
\item\
  \begin{framed}\noindent
    \cmd{Choose between}
    \begin{itemize}
    \item High Throughput: EOS mode 3, 100 keV, lens 8, from 2nA and above
    \item High Resolution: EOS mode 6, 100 keV, lens 5, 100-400 pA.
    \end{itemize}
  \end{framed}
\item \cmd{For  each option, available  Condition files will appear  in a table.   Choose the
    appropriate one.  If the desired condition file does not exist, ask Dr.  Shaikhaidarov to
    create it.}
\item \cmd{Press OK, and OK again}; %or do we have to press Apply?
\item \cmd{Tick the  "Restore" and "DEMAG" option.  Press "Check.."-button  to check that the
    condition file was created recently};
\item \cmd{\textbf{Manually change the aperture}:  Open the "Main Console Access"-door.  Pull
    and turn cylinder handle to the appropriate aperture};
\item \cmd{Press "Execute" $\rightarrow$ OK $\rightarrow$ OK. This will set the current};
\item \cmd{To  check the current,  choose FC (Faraday Cup)  in the "Stage  Control"-tab under
    Fixed Position , press MOVE}
\item \cmd{If the current is ok, move the Fixed Position to BE and press MOVE}
\end{itemize}
% add JEOL Aperture
% table %Create table for current and aperture


% choose current
\begin{table}[h]
  \begin{tabular}{|c|c|c|c|}
    & \textbf{Current} & \textbf{Aperture} &  \textbf{Lens}\\
    \textbf{High throughput} & 10-100nA & \iunit{300}{$\mu$m} - 8  & 4th Lens\\
    \textbf{High resolution} & 200pA & \iunit{60}{$\mu$m}  & 5th Lens
\end{tabular}
\end{table}

\subsection{Alignment after aperture is changed}
\label{Alignment}
\begin{itemize}
  \item \cmd{Choose BE as Fixed Position} (Press Regist. to remember the position). THis is the
  marker on the cassete that we will use for initial calibration;
\item \cmd{Press Move-button};
  \item Under the \mi{Image Control} heading, press SEM:
\item For faster scan \cmd{tick \mi{Rapid} and press \mi{Apply}};

\item \cmd{Adjust  Brightness and Contrast (Check  magnification).  It is a  good strategy to
    put the contrast to maximum, and adjust the Brightness};
  \item \red{Might be useful to click \texttt{x56}};
\item \cmd{Press \mi{WOBB} \ra Press on the lens you are using};
\item \cmd{Eliminate wobbling by turning dials on the machine stick - start with one to eliminate diagonal (the closest dieal) and then the other to eliminate other diagonal};
  \red{The image should pulsate but not move laterally.}
\item \cmd{Turn off wobbler, and wait for DEMAG to complete \ira SEM OFF}.
\end{itemize}

\subsection{Manual focus}
\label{subsec:manual-focus}

\begin{itemize}
\item Keep the SEM on and go to \texttt{BE};
\item Pull out the table and adjust focus;
  \item Press the \texttt{Stig} button and remove astigmatism.
\end{itemize}
\subsection{Calibration}

\subsubsection{Current calibration}
\label{sec:current-calibration}

\begin{itemize}
\item   \cmd{\mi{Calibration} button \ra In the \mi{Curr. Adjust.} tab}
\item Click ``Calibration''
  \item \red{Do no need to do the ticking and sections below most of the time};
\item In  the ``Curr.  Adjust.''tab,  tick Beam  Current Measurement, Beam  Current Alignment,
  DEMAG, and Beam Axis Alignment %CHECK what needs to be ticked here
\item Press "Execute".
\item Now,  the optical  beam alignment is  done and we  have to  recheck wobbling. To  do so
  repeat the steps described in sec.~\ref{Alignment}.
\end{itemize}

\begin{itemize}
\item   Untick    Beam   Current   Alignment    in   the
  "Curr. Adjust."-tab,
\item Execute \ira current should be read in the ``probe current'' box.
\item Save
\end{itemize}

\subsubsection{Standard mark detection calibration}
\label{sec:stand-mark-detect}

\begin{itemize}
\item Tick  AE and BE  Mark Detection in the  "STD. Mark
  Detection"-tab.
\item Execute
\item If
  \begin{itemize} \item  marks were found,  press "Save"
    $\rightarrow$ OK $\rightarrow$ OK.
  \item  If mark  detection  failed,  increase the  Scan
    width  to  40  $\mu$m  and  repeat  mark  detection,
    i.e.  Execute again,  and  if marks  are now  found,
    press "Save" $\rightarrow$ OK $\rightarrow$ OK. Then
    decrease  the  Scan  width  for   AE  and  BE  to  4
    $\mu$m.   Press   "Apply"   $\rightarrow$   OK   and
    Execute.  You are  now detecting  marks with  higher
    resolution. Press "Save".
  \end{itemize}
\end{itemize}

\subsection{Focusing}
\label{sec:focusing}

\begin{itemize}
\item   Tick   "Static    Focus   Correction"   in   the
  Focusing-tab.
\item Execute
\item Save
\end{itemize}

To correct for deflections

\begin{itemize}
\item   Tick  all   options   (Main   DEF.,  Sub   DEF.,
  Dist. corr.) in the "DEF Corr."-tab.
\item Execute
\item Press OK and "Save".
\item You may want to check that the calculations values
  for the convergence judgment X,Y are below 4 nm (6 nm)
  for lens 5 (4) in the "Log".
\end{itemize}

\subsection{Material Correction (aligning the sample)}
\label{sec:material-correction}

To correct for material and tell JEOL where to find the markers and chip markers. This ties in with Sec.~\ref{sec:jobmaker-1}.
\begin{itemize}
\item Tick ``Global Mark Detection'', ``Q  point'', ``Semiautomatic'', and ``Wafer'' in the ``Material
  Corr.''-tab.
\item Choose your wafer size and wafer window. %window C.
\item Enter your P and Q mark coordinates.
\item In  the "Global Mark"-tab,  press "RG  Detect Condition..." below  P Rough Scan,  and a
  window will open.  In the "Scan Type"-tab enter the  width of your mark. Do the  same for P
  Fine Scan Q Rough Scan, and Q Fine Scan.
\item Execute
\item Press OK, and move mark to the centre.
\item Press Continue
\item Save
\item Check P-point mark measurement result in the log-file after Global Mark Detection, take
  a note of the offset.
\end{itemize}
To detect chip marks
\begin{itemize}
\item In  the "Chip  Mark Detection"-tab, choose  Mode 1
  (detects  one  mark),  4   (detects  four  marks),  V1
  (virtual mark, height measurement in one point), or V4
  (virtual mark, height measurement in four points).
\item Enter coordinates of your mark(s).
\item Execute
\item Save
\item Update $\rightarrow$ Save $\rightarrow$ OK.
\end{itemize}

\subsection{JobMaker}
\label{sec:jobmaker-1}
\begin{itemize}
\item Select  Cassette name  and choose window. %add diagramm
\item Press Job  Property and a window  "Job 1 Property"
  will pop up.

\item In that window under the Calibration heading
  \begin{itemize}
  \item Select EOS Mode.
  \item Choose your Condition File.
  \item Choose your Calib. Menu, for example
    DIRE01. %Explain options
  \end{itemize}
\item Under the Exposure Condition heading
  \begin{itemize}
  \item Choose Scan Step
  \item Enter Beam Current
  \item Choose OL Aperture
  \item Choose dose (the resist does not matter)
  \end{itemize}
\item Under the Alignment heading
  \begin{itemize}
  \item Tick Global Mark Detection
  \item Choose Mode "Semi Auto"
  \item Copy your P and  Q mark coordinates from the Log
    file (You have  taken a note after  doing the Global
    Mark Detection).
  \item Tick Size
  \item Enter width and length of your marks.
  \item  Tick Height  Detection,  change  Offset to  the
    values found in the Log  file (from when Global Mark
    Detection was done)
  \item Tick Chip Mark Detection
  \item Choose Type
  \item Choose Error Mode
  \item Tick Height Detection
  \end{itemize}
\item Press  OK. The  window will close.  P and  Q marks
  appear on the wafer in the "EB Job Maker".
\item  Open  "ptnview"  on  the Desktop  to  check  your
  pattern.
\item If satisfied, return to "EB Job Maker".
\item  Below   Chip  List,   below  Chip  a,   click  on
  "..."-button. A window pops up.
  \begin{itemize}
  \item Choose your file and press Open.
  \end{itemize}
\item Press on "a"-button and drop it on the wafer.
\item Right click on it and select "Chip property".
  \begin{itemize}
  \item Choose Center Position
  \item   Choose   "Subfield  Sorting"   (Direction   of
    exposure)
  \item Press "Shot Rank Table..." next to Shot Rank and
    a  window will  pop up.  Here  you need  to set  the
    doses, enter Modulation in $\%$. Press Close.
  \item Enter your chip mark coordinates.
  \item Choose shape of your chip mark.
  \item Enter your chip mark size.
  \item Press OK.
  \end{itemize}
\item To  create an array,  right click on  your pattern
  and select "Array". A window pops up.
  \begin{itemize}
  \item Choose your array size (Number)
  \item Choose your chip to chip distance (Pitch)
  \item You may want to tick Grouping, especially if you
    are planning  to create  another array of  the array
    (Nesting).
  \item Press Ok
  \end{itemize}
\item You may create another array if you wish.
\item Save File.
\item Compile (triangle icon) and press OK.
\end{itemize}
Return to "jbx writer"
\begin{itemize}
\item Press "Exposure" button.
\item Press open and select your Magazine File, the file
  you have just compiled.
\item Press "Start Exposure"-button, press OK.
\item You can monitor,  the progress of exposure: Yellow
  (writing), Blue (completed)
\end{itemize}

\subsection{For JEOL engineers and advanced users only}
% \iframe{RedHat Enterprise Linux 6}

\subsubsection{Focusing}
\begin{itemize}
\item Tick  "AE Mark Detection" and  "BE Mark Detection"
  and execute;
\item  If it  fails  press  \mi{AE Detect  Condition...}
  (\mi{BE  Detect Condition...})  in the  to check  mark
  detection parameters \ra \cmd{Increase the Scan widths
    to 100 $\mu$m for example};
\item If it  still fails, double check  that the wobbler
  has been turned off.
\item  If wobbler  was turned  off but  it still  fails,
  change  position,  tick AGC  (make  sure  the mark  is
  positioned in the centre) and execute
\item If  it still  fails do a  manual scan  by pressing
  \mi{Wave} under the \mi{Image Control} heading.
  \begin{itemize}
  \item In the \mi{Wave}-tab change Gain and Offset;
  \item  Press \mi{Applies  to Calibration}  \ra \mi{All
      select} $\rightarrow$ Apply;
  \item Stop Wave and try executing again;
  \end{itemize}
\item Save $\rightarrow$ OK
\item Untick AGC, decrease  scan width to 4$\mu$m, Press
  Apply $\rightarrow$ OK
\item Execute \ra Save;
\end{itemize}
In the \mi{Focusing} tab
\begin{itemize}
\item Only tick \mi{Static Focus Correction}
\item Choose SF/SSX/SSY;
\item Press Execute \ra Save;
\end{itemize}
In the "DEF. Corr."-tab
\begin{itemize}
\item   Tick   \mi{Main    DEF},   \mi{Sub   DEF},   and
  \mi{Dist. Corr.};
\item Allowable gain error should be 4nm;
\item Distortion  error should be  6nm (4nm) for  lens 4
  (5);
\item Execute $\rightarrow$ OK $\rightarrow$ Save;
\end{itemize}

\subsubsection{Checking how marks are detected (this does not affec the calibration file)}
In    the   \mi{Material    Corr.}-tab   \ra    \cmd{use
  \mi{semi-automatic}   mode}.    \textbf{First   Global
  Marks:}
\begin{itemize}
\item Have only \mi{Global Mark Detection} ticked
\item  In the  \mi{Global Mark}  tab have  only Q  point
  ticked;
\item Enter the position of your Q mark of your specific
  wafer;
\item Execute $\rightarrow$ Ok;
\item If it fails, untick  Rapid, increase scan width to
  maximum and press Apply;
\item When  you found the  markers, tick Grid  under the
  SEM-tab  and  press  apply,  position  marker  to  the
  centre, and press Continue and Save;
\item In the \mi{Log} check  the offset and type it into
  the Offset field in the Global Mark tab
\item Save \iframe{\red{Here  take note of the  P, Q and
      offset values. They will be used below.}}
\end{itemize}
\textbf{Then Chip Marks:}
\begin{itemize}
\item Have only \mi{Chip Mark Detection} ticked;
\item Choose a mode;
  \begin{itemize}
  \item 1 for one mark and offset, but no rotation;
  \item 4 for four marks and rotation, measures position
    and height;
  \item V1 and V4 for auto focus
  \end{itemize}
\item Execute \ra Update $\rightarrow$ Save
\end{itemize}

 \subsection{Creating pattern\label{subsec:jobMaker}}
 Claibrating  the  EBL is  all  good.  We also  need  to
 prepare the  pattern that  it will  expose. Do  this in
 \verb|EB Job Maker|. The main goals are:
 \begin{enumerate}
 \item Define the coordinates of the global markers:
   \begin{itemize}
   \item \mi{Job property};
   \item Select Window position on the sample holder you
     are using (A, B, C ,D);
   \item Select the window size. Big wafers are 3" and a
     2x2 chip fits a 2";
   \item Define the coordinates  of the global markers P
     and Q  \red{using the coordinate and  offset values
       you  found during  calibration. Yes  - just  copy
       them.}
   \item  \cmd{Tick other  properties  such as  current,
       calibration file etc.}
   \end{itemize}
 \item Select the .v30 pattern to expose \ra Right click
   to make an array of them;
 \item \cmd{Right click on  the pattern} \ra \cmd{Define
     the  position  of  the  chip  mark  for  the  first
     pattern. If you want to offset the pattern to be in
     betwen the chips \red{apply (-402,-460) offset}};
 \item \cmd{Save and load it up for exposure.}
 \end{enumerate}


 \begin{framed}\noindent
   For movement, make sure that \mi{cassette} in the top right corner is the one you chose.
 \end{framed}
 \newpage

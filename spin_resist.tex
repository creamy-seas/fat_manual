% -*- TeX-master: "../fat_manual.tex" -*-


\section{Spinning resist\label{sec:ebeam}}

\begin{itemize}
\item \cmd{Turn on the vacuum buttons};
\item \cmd{Select program and position wafer};
\item \cmd{\red{Do not touch the metallic buttons} \ira click on  the screen pump to activate it - left button on screen
    (green and shows injection)};
\item \cmd{Click green arrow to start spinning (red stop sign to stop)};
\item \cmd{Once spinning is finished, press the leftmost button on screen to deactivate vacuum};
\item \cmd{\red{Turn off the pump with the switch under the fume cupboard}}
\end{itemize}

\begin{framed}\noindent
  \red{Clean samples  with acetone$^o$C  for 10  minutes, and  then wash with  isopropanol (spray  and spray  again with
    pipette). Wash things with the water gun. Don't need to wash isopropanol dishes.}

  IPA = 2-propanol = Isopropanol
\end{framed}

\begin{framed}\noindent
  In case of contamination, wash off with relevant resist remover (AR 600-71 for ARP6200 for example).
\end{framed}

\subsection{Placing resist}
\begin{enumerate}
\item Get the sample, clean in propanol and dry with air gun.
\item Turn on baking tray (e.g.  170$^o$C), by holding down \textbf{set} and turning dial.
\item Choose correct program (e.g 4500rpm) on the \textbf{spin controller} \ira \cmd{D \ira E \ira E \ira \quote{Program
      number} \ira E}. Program numbers are deciphered on the tissue paper next to the machine.
\item Place  sample on the  spinner.  Ensure that  a suitable  \textbf{base} has been  chosen - not  too big or  small -
  \red{the sample must fully cover the base};
\item Put on the  resist.  For the 2D flakes, we used  copolymer 13\%/PMMA 4\%.  For other ones we  used ZEP.  Use clean
  syringe.  \red{Do not touch the end  of the pipette. Do not go into the solution multiple  times - take what is needed
    in one move!}
\item \red{Turn on the black plug, so that the pump sucks the sample to the base!}
\item Press the green button to begin the two minute cycle.
\item Place on the baking tray and cover with a foil hat.  Time for 10 minutes.
\item \red{In the meantime clean the base.  Acetone on a tissue or in a dish and put the base in}
\item \red{Put on a healthy scratch for focusing in SEM.}  Check with optical microscope and remember orientation.
\item \red{Turn off the heater, the rotator.}
\end{enumerate}

Proceed to SEM, in order to perform e-beam lithography. Once  complete, its time for development, to cut out the exposed
holes in the resist.

\subsection{Developing}
\begin{enumerate}
\item Dunk in developer, which varies.  For the  2D flake sample, we used T(Toulene):IPA(Isopropanol=2-propanol) 1:3 for
  10  seconds, then  clean with  \red{isopropanol} to  remove first  layer.  Then,  7\% H20  in IPA  for 20  seconds and
  isopropanol to develop the second layer.
\item \red{Acetone will remove all of the layers of resist}
\end{enumerate}

Then one proceeds  to evaporation.  This can  be done on the  Plassys (Sec.\ref{sec:plassys}) or on  the Edwards machine
(Sec.\ref{sec:edwards}). One  deposits a film, after  which lift off, of  the remaining resist is  performed.  For flake
contacts we did 10nm titanium and 80nm aluminium

\subsection{Lift off}
\begin{enumerate}
\item To remove aluminium, use photoresist developer 1min and rinse with water.
\item To remove ZEP, use pixilen for 30sec.
\item \red{To remove anything use  acetone.  Put the dish with a lid to bake for  10 minutes. Everything should come off
    by itself.}
\item Rinse in isopropanol.
\item Look in microscope for result.
\end{enumerate}
\newpage

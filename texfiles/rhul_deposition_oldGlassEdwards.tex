\section{RHUL Oled Edwards depositer\label{sec:edwards}}
\begin{framed}\noindent
  \begin{itemize}
  \item \red{Remember to follow the  arrow on the handle which points
      to current pump configuration}
  \item   \red{If  thickness   monitor   is   not  responding   click
      \cmd{shutter}   button   mutliple    times   until   it   shows
      \cmd{close}.}
  \end{itemize}
\end{framed}

\subsection*{Starting up}
\begin{itemize}
\item \cmd{Turn  on the Edwards Plug  \ira open front door  and press
    \cmd{Green button}} (its the breaker);
\item \cmd{Turn on the pump with in a switch in the back corridor};
\end{itemize}

\subsection{Preparation}
\begin{enumerate}
\item \cmd{Turn on  diffusion pump} (yellow handle).  It  needs to be
  constantly pumped by another pump (we never turn it off. if needed,
  pull out its plug). Give 20 mintues to heat up;
\item \cmd{Air admit}  to vent chamber \ira red  indicator lamp turns
  on \ira \cmd{wait 2 min};
\item \cmd{Remove the  two bubble cases \ira \red{air  admit to off}}
  in order to not forget afterwards;
\item \cmd{Remove  chimney \ira If  required unscrew the  plasma etch
    setup.} There  are two fairly large  screws on the bottom  of the
  holding   plate   and   two   screws   next   to   the   protractor
  dial. \cmd{Fully unscrew them and place  to one side \ira shfit the
    plasma etching setup (needle, protractor, base) to side of hole};
\item \cmd{Screw in the sample holder to  a bar that is placed on top
    of the hole};
\item \textbf{For spiral} \cmd{get 10cm of the deposited material and
    wind it onto the spiral coil, ensureing good contact};
\item \textbf{For boat} \cmd{Top up material};
\item There are three electrodes  in the machine.  Connect the spiral
  and boar, one to common, one to one of the leftover electrodes
  \begin{itemize}
  \item \red{The far back electrode is ground = common}
  \item Side one is 1;
  \item Near-back one is 2;
  \end{itemize}
\item \cmd{\red{Place separator between boat and spiral}.} It will be
  lifted up when the diffusion pump  is activated and the stage rises
  up;
\item \cmd{\red{MAKE  SURE THAT  THE CRYSTAL  USAGE ON  THE THICKNESS
      MONITOR IS NO MORE THAN 80}};
\item \cmd{Place chimney \red{so that  thickness monitor can see both
      sources}} \ira \cmd{\red{ensure that shutter is closed}};
\item \cmd{Place any wiring on the top stage};
\item  \cmd{Position  lids with  exclamation  sign  opposite a  small
    circular pin};
\end{enumerate}

\subsection{Pumping}
\begin{enumerate}
\item  \cmd{Turn  lever  \red{anticlockwise} to  roughing}  to  begin
  pumping \ira check  the \cmd{pirani 10} and \cmd{black  mks box} on
  other depositor for pressure;
\item \cmd{Turn the ``DO NOT USE FORCE'' pin out} - this relieves the
  pressure in the gas lines so  that there is no leakage (controllers
  on the wall produce a very good seal);
\item Add 3 scoops of nitrogen to trap oil;
\item \cmd{Once \red{200\,mTorr}  shows on the mks  barometer and the
    ``PIRANI  10''  (around  5min)  \ira  handle  \red{clockwise  360
      degrees} to hook up diffusion pump};
\item \cmd{Turn on Penning 8};
\item  \red{\cmd{If   the  ``OERNNING   VALVE''  doesn't   rush  over
      $ 10^{-4} $ then refill the nitrogen trap}};
\item  \cmd{Keep  track  of  ``PENNING  VALVE''  \ira  \red{wait  for
      $8 * 10^{-6} torr $} which takes around 1 hour};
\end{enumerate}

\subsection{Depositing}
\begin{enumerate}
\item \cmd{On the  thickness block ``LAYER'' \ira  use ``INC/DEC'' to
    choose the layer}.  Make sure correct ``DENSITY'' and ``Z-VALUE''
  show up;
\item \cmd{Turn the black lever  to \quote{source 1} or \quote{source
      2} (horizontal positions), depending on what we are going to be
    depositing};
\item \cmd{Press  yellow \quote{LT}  (low voltage) button  \ira press
    \quote{green} safety  button \ira start ramping  the current up.}
  Typically it would reach 20-30\% for good deposition;
\item \cmd{\red{May  need to turn  handle anticlockwise to  lower the
      stage - the separator might be blocking even dispersion!}}
\item  \cmd{Open  shutter and  press  \quote{open}  on the  thickness
    monitor.  Once  required thickness is reached,  close the shutter
    manually};
\item \cmd{Ramp  the current down  to 0 \ira Press  red \quote{reset}
    button, so that the green button jumps up \ira turn LT off};
\item \cmd{\red{Only after this, may you switch the voltage source}};
\end{enumerate}

\subsection{Final steps}
\begin{enumerate}
\item  \cmd{Turn handle  to \quote{backing}  until it  won't turn  no
    more}  and  wait for  5  minutes  to  allow  stuff to  cool  \ira
  \cmd{turn off the penning monitor};
\item \cmd{Turn the \quote{do not  use force} knob inwards, to cancel
    the atmpshpehric pressure;}
\item \cmd{Press \quote{air admit}} to vent the chamber;
\item \begin{itemize}
  \item Remove sample and divider (if used);
  \item Remove boat and spiral;
  \item Rescrew the  plasma setup \ira put on glass  cylinder and top
    bit;
  \item Put chimney back in;
  \item Replace bubble and imposion guard.
  \end{itemize}
\item \cmd{\red{Press \quote{air admit}}};
\item \cmd{Handle  to \quote{roughing}, wait for  200mTorr reading on
    mks} \ira \red{DO NOT LEAVE LIKE THIS!!};
\item \cmd{Turn off diffusion pump};
\item \cmd{\red{Turn handle to backing position and leave!}}
\end{enumerate}


\newpage

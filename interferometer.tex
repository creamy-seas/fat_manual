\section{Interferometer}
  Interferometer is used to evaluate the frequency of an unknown source. This is done by splitting up a beam with a beam splitter, sending two components down different routes with a path difference $ 2\Delta L $. Thus, each time 2$ \Delta L =\equiv m\lambda\Rightarrow \lambda = 2\Delta L/m$, there we would have undergone a full interference pattern oscillation, from maxima to minima back to maxima. Thus the distance moved by the mirror, $ \Delta L $, to complete one period of the interference pattern is used to evaluate the frequency of the source.
  
  \begin{enumerate}
  	\item \cmd{Set up mirrors, beam splitter, detector.} Remember that the mirrors have a focal length - find it by finding where light reflecting of the mirror gets the most intense. This is the distance at which the source of radiation must be placed relative to the `guiding' mirror of the interferometer (which directs the source to the beam splitter);
  	\item \cmd{Adjust the `guiding' mirror so that the laser light falls directly in the center of the beam splitter and `stationary' mirror \ira Adjust the `stationary' mirror so that this reflected right coincides with the incident one on the beam splitter;}
  	\item \cmd{Adjust the `moveable' mirror so that both reflected paths cross in a single point;}
  	\item Can use a tissue paper to see interference pattern at the output. Put the detector somewhere there.
  \end{enumerate}  
 \newpage
 
 
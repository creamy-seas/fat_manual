% -*- TeX-master: "../fat_manual.tex" -*-

\newpage\section{EBL RHUL File Preparation (Beamer)
  \label{makeFile}}
For making the pattern we can:
\begin{itemize}
\item Crete a full wafer design in autocad/beamer  \ra supply the P, Q coordinates \ra perform
  a full exposure;
\item Design individual pattern in autocad/beamer \ra  supply P, Q, chip marks, arrays at JEOL
  (see Sec.~\ref{subsec:jobMaker}) \ra perform exposure.
\end{itemize}
  
When creating a full wafer pattern, we will use 1/4 of a 3" wafer \red{which we load into a 2"
  windows}
  
 \subsection{Autocad}
 To make patterns for the lithographer (optical and electron beam) the designs need to be made
 in Autocad. Draw whatever you want, but follow these rules:
  
 \begin{itemize}
 \item There must be an object in the zero layer;
 \item All shapes \textbf{must} be polylines. They must be joined up in a single line, or they
   wont work;
 \item Units are in microns;
 \item \red{Place small markers in the corners of the pattern so that centering is correct};
 \item Once design is finished, \texttt{select all \ra  Purge \ra All \ra type "none*" \ra Yes
     to all};
 \item Save as \texttt{Autocad DXF} to use it in beamer.
 \end{itemize}

 \begin{framed}\noindent
   Important is the \cmd{extent} of the pattern. This is the dimensions that the pattern
   will be overlayed on e.g.  $ 100 \times 100\,\mu$m which we can enlarge, scale etc.
 
 \begin{center}
   \red{The  base  dose,  relative  to  which  all  the  dosing  is  done,  will  need  to  be
     experimentally found  by doing test  patterns on JEOL.  Only  then do the  relative doses
     that the program makes will mean something.}
 \end{center}
 \end{framed}
 
\subsection{General information}
\begin{itemize}
\item Sending is done via \verb|FTP| with \texttt{FileZilla};
\item Internally, beamer uses its own file format during its processing;
\item Double  click in bottom  right symbol  to view pattern.   \cmd{Right mouse click}  to do
  measurements. \micmd{Tick} and \micmd{clear} to  retain commands during movement and zooming
  in;
\item \micmd{Shot settings} button to see dots that JEOL will fire;
\item \micmd{Layout B(oundary)box} button to show extent of pattern;
\item \cmd{Colour by cell} to highlight colours in design;
\item \mi{Doses \ra Get Limits from Layout} to recalibrate the dose scale;
\end{itemize}

\subsection{How exposure is being done}
\begin{itemize}
\item \verb|JEOL| switches between different fields, `stitching' them together;
\item  JEOL starts  in the  upper  left.  Then  it  fills either  $ X  $  or $  Y $  direction
  trapezoids;
\item \textbf{Shot  size} =  size of  the ciruclar  shots that are  going to  be taken  by the
  Gaussian beam.  Typical is 2-3\,nm, defined by the regime of JEOL;
\item  \textbf{Shot  pitch}  =  beam  step  size  as  it  moves  from  position  to  position;
  \ipic{5cm}{jeol_shot}
\item Designs snap onto JEOL grid, which could distort shapes.
  % \item Field changes from 1\,mm to 100\,$ \mu $m,
  %   depending on how little stage movement you
  % 	 want, and how accurate you want the exposure;
\item Backscattering 35\,$ \mu $m for 100\,keV and silicon;
\end{itemize}
\newpage
\begin{center}
  {\Huge \textbf{BEAMER TIME}!}
\end{center}
\begin{framed}\noindent
  Beamer works in relative doses:
  \begin{itemize}
  \item 1 = no dose modulation;
  \item 0.8 = -20\% of the dose;
  \item 1.2 = + 20\% of the dose;
  \end{itemize}
    
  \red{\textbf{The final dose depends on your shape, voltage, and the shot pitch}}

  To use variables in a for loop, declare them in the transform function (for example) instead
  of a number

  \begin{equation}
    \label{eq:variable}
    Rotate:    \%VARIABLE-NAME\%
  \end{equation}

  \noindent and then increment them in the \textbf{loop} function.
\end{framed}

\subsection{Import}
\begin{itemize}
\item Eats \cmd{dwg} \ira converted to internal file format;
\item \micmd{Convert  colour to  datatype} to  import layers  by colour  \hfill \red{I  do not
    use}\ec;
% \item \micmd{Import zero  width} to import lines and  convert them to a certain  width so that
  % they can be traced out;
% \item \micmd{Extraction  type} to choose  whether to extract  full pattern or  small sections.
  % \iframe{\textbf{\red{The selected area is known as the extent}}}
\item The pattern is centered relative to the most extereme elements;
  % \item You can also choose \mi{cell} i.e. select out
  %   the sublocks from which the design is made;
  % \item Cells work by applying an array pattern to
  %   these individial cells.  THis is heirachial
  %   structure.
\end{itemize}

\subsection{Actions}
\begin{itemize}
\item \textbf{Edit} to allow change of  design. \red{\textbf{Run it standalone to create empty
      design}};
\item \textbf{Mapping} change layer names to new ones (or into other layers);
\item \textbf{Heal} removes overlaps - \red{\textbf{merges into 1 layer}};
\item \textbf{Merge} collect different layers;
\item \textbf{Transform} to  position about a chosen  origin e.g. set \mi{center}  to 0,0. The
  pattern is done in order of appearance;
\item \textbf{Grid} to put the design on a grid with new dimensions;
\item \textbf{P-XOR}  Delete odd  number of overlays  - \red{can be  used to  check difference
    between original and processed pattern};
\item \textbf{Loop} to repeat the operations.  Merge results and \mi{heal} to remove overlap;
\item \textbf{FDA} modulate DOSE by layer;
  % \item \textbf{Bias} - sizing of elements;
\item \textbf{Bias} is used to trim based on edges. Can be used to remove thin structures;
\item \textbf{PEC} is standard proximity correction;
\item \textbf{Ebeam simulation} is to see how the beam will write;
\item \textbf{Visual Job} is a standalone block which we run independently.  It creates a text
  file that specifies how the generate the arrays, maybe some scaling of the dose;
\item  \cmd{Split tool}  to branch  out lines;  \iframe{\textbf{To draw  line }\newline  draw it  \ra
    \mi{export\ra advanced} and choose \micmd{reserved line class}.}
\item \textbf{Fracture} how the design is broken up.  This is passed onto proximity correction
  (which subfractures the fractures)
\item  \textbf{Fracture  \iRa Advanced}  \ira  (\red{\textbf{see  below}}\ec Combine  floating
  (fields are  freely positioned)  and fixed  (fields are in  a grid)  and apply  to different
  layers. Select all layers with *);
\end{itemize}
\subsection{Export}
\begin{enumerate}
\item \textbf{Comma seaparated list} to choose layers to extract.
\item \textbf{Machine type} \cmd{JBX-81000FS};
\item \textbf{File type} is \cmd{.v30} (changes resolution and field size automatically)
  \begin{itemize}
  \item \cmd{3 - 100kV high Throughput} \hfill \red{maximum field size is 1mm$ \times1$mm};
  \item  \cmd{6  -   100kV  high  resolution}  \hfill  \red{maximum  field   size  is  100$  \mu
      $m$ \times100\mu $m};
  \end{itemize}
\item \textbf{Shot settings} \hfill \red{\textbf{finely spaced shots $ \equiv $ small doses on each
      shot}}.
  \begin{itemize}
  \item  \cmd{Pattern unit  (resolution)}  (depends on  the  mode, 3  or  6, \textbf{fixed  by
      JEOL}). \red{Essentially it is the resolution of the grid on which the shots are made}
  \item \cmd{Shot pitch (step size)} is how much to separate the shots on this grid;
  \end{itemize}
  \red{\textbf{Make  sure you  keep the  same shot  pitch in  the JEOL  machine!  JEOL  should
      prevent incorrect pitches.}}
\item \textbf{Field settings (JEOL moves stage to expose each field)}
  \begin{itemize}
  \item \textbf{Size} of each  field e.g.  1\,mm $ \times $1\,mm is \textbf{fixed  by JEOL} - below
    we  change  the  data  within  the  1\,mm fields,  putting  and  taking  away  structures,
    overlapping and other shit;
  \item \mi{Advanced \ra fixed} to position fields like a grid;
  \item \mi{Advanced  \ra floating field}  to look  for large blocks  of items and  center the
    fields on them;
  \item \mi{Advanced \ra  manual \ra view layout} to define  you own field.  \textbf{Selecting
      when floating} \mi{Shift \ra draw box \ra  double click}.  \red{Make sure box is smaller
      than the main field};
  \item \mi{Traversal path} to see how exposure will jump between the fields;
  \item \mi{Center to field} to make sure that important elements are in the field center.
  \end{itemize}
\item \cmd{Shot pitch factoring}  is where you tell the shots to be  perfect on the edges, and
  overlap in the middle of the structure where it is not important; \ipic{5cm}{jeol_pack}
\item \cmd{Feature  sorting in field}  how to  save the order  of elements inside  the fields;
  \iframe{\micmd{Feature sorting in field} \ra
    \begin{itemize}
    \item \textbf{By geometry} to fill objects wihtout jumping or another option;
    \item \textbf{Left to right};
    \item \textbf{By layer};
    \item \textbf{Writing regions} fills in region by region, who size you set.
    \end{itemize}}
\end{enumerate}

  \subsubsection{Multipass Tab}
  This is  where you say  for the beam  to pass  multiple times across  the same areas  to fix
  stitching problems
  \begin{enumerate}
  \item \cmd{Field overlap},  so that structures on  the edges of two fields  are not directly
    cut   along  the   field   lines,  but   sorted   into  the   left   and  right   sectors;
    \ipic{5cm}{jeolOverlap}
    % \item \textbf{For the above procedure use}
    %   \cmd{share between fields} option;
    % \item \cmd{Interleaving fields} are two combs
    %   which feed into one another, so that you can
    %   definitely connect ohmic contacts across two;
  \item \cmd{Multipass shifting and  lowering dose} is when a pattern broken up  in such a way
    that the pattern is exposed 4 times, and given 1/4 of the dose each time. Stitches between
    fields   are  smeared   out  (as   they  are   1/4  intensity   instead  of   full  dose).
    \ipic{6cm}{stiching_multipass}
    \begin{itemize}
    \item Select number of passes and the overlap in terms of the mainfield dimensions. 0.5 is
      a good value.
    \item \red{Remember the size of the subfield (sub  part of the mainfield) should not be an
        factor of the mainfield size, or periodicity will appear at integer multiples.}
    \end{itemize}
  \end{enumerate}

  \subsubsection{Improve field movement}
  \label{sec:impr-field-movem}
  For improved movement of microscope do the following:
  \begin{enumerate}
  \item Get pattern and \textbf{bias} it to grow a rectangle around it;
  \item \textbf{Merge} to the original pattern -  now you have a rectangle layer positioned on
    top of your pattern.
  \item In export,  select \red{\textbf{region-layer}} to choose the layers  that will specify
    the centers of the respective fields.  The machine will move to those rectangles, and look
    for exposure patterns there.
  \end{enumerate}

 \subsection{Increasing speed}
 To increase  speed, we need to  draw the outline accurately,  and do the filling  at a larger
 current. Thus we need an outline and a fill design:
  
 \begin{itemize}
 \item \cmd{Bias} function will take your pattern and trim (e.g. 100\,nm) from all edges;
 \item \cmd{MINUS}  this pattern  from the  original \ra you  get an  outline of  the original
   pattern;
 \item Then take another \cmd{bias} and add (e.g.  50\,nm) to the first bias to create overlap
   between the cut out bias pattern (outline) and the middle pattern.  \ipic{7cm}{jeol_frame}
 \end{itemize}

 \subsection{Proximity correction}
 \red{Uses \cmd{Tracer} which we do not have}
 
 \begin{itemize}
   % \item The proximity correction functions are in
   %   \mi{GenSys \ra Tracer \ra 2DArchive};
 \item \red{Long range correction is always being done!}  Pixel based approach. It measure the
   local density,  in the range  of $ \beta  $ and performs a  correction based on  this extra
   exposed dose;
 \item \micmd{Archive} to select
   \begin{enumerate}
   \item Substrate e.g. Silicon+200nm PMMA;
   \item Z-position to set the position where the  correction should be the best (could be top
     of bottom of resist).  \red{\textbf{0 for top of resist}}
   \end{enumerate}
 \item \cmd{``Accuracy''} should be 1\%.
 \item Tick \micmd{Include Short Range Correction}  for small structures $ \sim 30\,$nm, where
   the size of the onset beam can affect neighboring patterns.  Off by default;
 \item \micmd{Effective short-range dose};
 \item   \micmd{Maximum   number    of   dose   classes}   is   the    number   of   different
   doses. \micmd{Accuracy}  sets the accuracy  that are given to  the dose classes.   Too much
   accuracy  fractures the  design to  assign these  different doses,  \textbf{\red{so do  not
       overdo it}};
 \item \micmd{Minimum dose} exists  so that beam does not jump too fast  (low dose, means beam
   can move across this area very  fast); \iframe{\micmd{Isodose grid} defined the fidelity of
     the grid to be used \red{\ra \textbf{make sure that the \micmd{Shot Pitch} is larger than
         the grid  so that you  don't have  10\,nm fractures, which  you fill with  6\,nm shot
         steps}}}
 \item \micmd{Minimum figure size} sest the finest fracture that can be made.
 \item \mi{Examples} \ra  \verb|TenDoses_100um_200um.fbd| that allows to expose a  set of test
   patterns, which can be used to calibrate  you own material (for future proximity correction
   values).
 \end{itemize}

 \subsection{Visual-Job - prepare for exposure }
 \label{sec:visual-job-add}

 To  open,  click on  the  icon  in  the toolbar  in  BEAMER  (multiple dose  tables  created)
 \red{\textbf{or use the CHIP PLACE COMMAND} (single dose table created)}
 \begin{enumerate}
 \item Set the marks;
 \item Create job name - \red{\textbf{must be capital letters}};
 \item \textbf{Shot pitch};
 \item \textbf{Array}
   \begin{enumerate}
   \item Set the extent of the pattern slightly larger;
   \item Center the pattern;
   \item \textbf{Array  dose} (start  + end e.g.   0.8 to 1.2,  linear) -  separate \cmd{.jdi}
     files are created;
   \item Open \mi{Place data in the side menu bar};
   \item \textbf{Pattern Data} to set the marker.
   \item \mi{Assign} to put the marker on the pattern.
   \end{enumerate}
 \end{enumerate}

 \newpage
 \subsection{Compiling on RedHat}
 \begin{center}
   \begin{itemize}
   \item {\cmd{v30} files are created in \cmd{beamer} and contain the \textbf{geometry} of the
       pattern;}
   \item {\cmd{jdi} files  are created in \cmd{beamer} and contain  the \textbf{doses} for the
       pattern;}
   \item {\cmd{jdf}  files are created in  \cmd{Jod Editor} and create  \textbf{arrays} of v30
       patterns and \textbf{assign doses} to them.}
   \end{itemize}
 \end{center}
 \vspace{1cm}
 \begin{enumerate}
 \item Create an jeol exposure program in \mi{\cmd{Jod Editor}} \ra \cmd{\mi{File \ra Save As}
     \ra} save this project as a folder;
 \item Copy the \mi{\cmd{.jdi}} files into this folder;
 \item Open \mi{\cmd{terminal}} and run the command
   \begin{center}
     \cmd{./jeolrhul}
   \end{center}
 \item Navigate to the jeol exposure program folder created using \cmd{cd} e.g.
   \begin{center}
     \cmd{cd ilya/2018files/transmonnpc}
   \end{center}
 \item Run a script to apply a modulation of +0\% $ \cdots$+100\% of the dose in the \cmd{.jdi
     file} across: \iframe{\[
       \begin{aligned}
         \parbox{6cm}{\color{gray} Array number 2 \newline(as it appears in the jdf file)}& \quad  & &\text{\cmd{jeolrhul\_modulate\_array} \quad 2 \quad 100 \quad doseFile.jdi}^{*}\\\\
         \parbox{6cm}{\color{gray} All arrays that use pattern A \newline (as set in \mi{Jod Editor})}& & &\text{\cmd{jeolrhul\_modulate\_pattern} \quad A \quad 100 \quad doseFile.jdi}^{*}\\
       \end{aligned}
     \]
     \hfill The specifier $ ^{*} $ is only needed if there are multiple jdi files files in the
     folder.  }
	
 \item  Compile   the  program  in   \mi{\cmd{Jod  Editor}}   by  clicking  the   black  arrow
   $  \blacktriangleright  $\newline  \red{\textbf{Do  not  save} as  it  will  overwrite  the
     \cmd{.jdf} file.}
 \end{enumerate}

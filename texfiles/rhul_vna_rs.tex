% -*- TeX-master: "../fat_manual.tex" -*-

\section{RHUL VNA Rohde and Schwartz}
\begin{itemize}
\item \texttt{Sweep} \ra \mi{sweepTIme}: continuous = sweep, CW = single freq
\item \texttt{Scale} \ra autoscale
\item \texttt{Span} \ra Center 10GHz, span 16 GHz;
\item \mi{Power BW AVG} to set power, bandwidth;
\item RF ON.
\item \red{When  first connected via  ethernet, run the \mi{rsvna-lv\_2\_42\_0}  installer and
    run the \mi{Set OPC  timeout.vi} with 300000ms = 300 seconds or larger  in the program, so
    that the wait time from the VNA to PC is increased.}
\end{itemize}
 
\paragraph{VNA} compares  the \textbf{phase} and  \textbf{amplitude} of an outgoing  signal to
the one fed into the system.
 
 \begin{figure}[h]
   \ifigure{4cm}{vna}
 \end{figure}

 \section{Trigger}
 \label{sec:trigger}

 For two tone measurement set \red{Trigger from sync}

 For Rabi set \red{Free run}
 
 \subsection{Correct phase}
 \label{sec:correct-phase}

 If the phase is choppy, then do \cmd{click \quote{Offset-embed} \ira \quote{Auto-length}}.

 \subsection{Measure Q factor}
 \label{sec:measure-q-factor}

 To measure q factor of resonator:
 \begin{enumerate}
 \item \cmd{Click \quote{Marker} and select 4 markers};
 \item \cmd{Click \quote{Band filter} \ira  \quote{Bandpass reference to max}.} \red{make sure
     it's set to 3dB};
 \item Markers  will be  moved to positions  where the  peak falls off  by 3dB  (equivalent to
   halving) to work our the Q factor
 \end{enumerate}
 \newpage
 
\section{C++ notes and reminders}
\red{Beware of integer  multiplication - it can easily go  out or range
  and give a negative number! Make sure you covnert to doubles first}

\begin{itemize}
\item To create lib file from a def file:
  \begin{center}
    \texttt{Visual studio command prompt  (an application) \ira lib.exe
      /def:foo.def};
  \end{center}
\item ADQAPI.dll contains commands for execution;
\item ADQAPI.h contains the function prototypes;
\end{itemize}

\section{Threading}
\begin{itemize}
\item 4-core computer downstairs;
\item Do not let threads access the same variables;
\end{itemize}

\section{Project Compilation}
\subsection{Compilation}
\begin{enumerate}
\item Create the project;
\item   Copy   \cmd{ADQAPI.lib,    AQQAPI.h,   fftw3,   libfftw3-3.dll,
    libfftw3-3.lib} to  ./lib and add them  to the relevant lib  tab in
  solutions explorer;
\item Right  click on the project  in the sidebar \ira  properties \ira
  linker  \ira input  and enter  \cmd{ADQAPI.lib} into  \cmd{additional
    dependencies};
\item Add \cmd{\#include ``ADQAPI.h"/``fftw3.h''} into the cpp file you
  need them in;
\item  Next  to  \cmd{debug  \ira  Solution platform}  to  x64  or  x86
  depending on  the library being  used.  x86  is good enough  for most
  parts.
\item \cmd{General \ira Output Directory \ira \quote{\$(ProjectDir)bin}}
\item     \cmd{General     \ira    Intermiediate     Directory     \ira
    \quote{\$(ProjectDir)build}}
\item \red{To run a  DLL (C++ or python) it needs to  be in the project
    folder.}   Project options  \ira  Debugging  \ira Enivronment  \ira
  \cmd{PATH=\%PATH\%;\$(ProjectDir)lib}.
\end{enumerate}
\subsection{Linking External Libraries}
\begin{itemize}
\item A static library (.lib) is copied fully into program;
\item A  program that uses  a DLL library  (.dll), will only  load it's
  functions at run  time.  For this to happen, the  dll file must exist
  at run time:
  \begin{itemize}
  \item It still requires a .lib file;
  \item This lib file contains  wrapper functions, that the linker puts
    into the main program.  It's fast for the linker to do this;
  \item At  run time function  called ->  stub found in  .lib (function
    calling function from dll) \ira dll function called;
    \begin{framed}\noindent
      \texttt{Linker \ira General \ira Additional Library Directories =
        \$(ProjectDir)lib} \vspace{1em}

      \texttt{Input -> Additional Dependencies -> libfftw3.lib}
    \end{framed}
  \end{itemize}
\end{itemize}

\subsection{Important Errors with Digitiser}
\begin{itemize}
\item Use ADC capture lab to connect to device. If in the log window it
  says \quote{Device not found}.

  \cmd{Turn off  the computer the PXIe-1065  box. Turn the box  on \ira
    turn computer on};
\item Do  NOT run ADCaptureLab  and attempt  to communicate via  C++ or
  something;
\item Once buffers have been allocated, make sure that for GetData, the
  third argument number of bytes per buffer element, must be set to the
  type size of the buffer. For int  it should be 4. For short it should
  be 2;
\item LabView operates in 32 bit mode (i.e. x86, and therfore DLLs made
  for it must be  run in x86 debug mode + need to  use the 32bit ADQAPI
  libraries);
\item Calibration  is $  2^{16} $  levels/2.2V =  29789.1levels/Volt so
  resolution is 33$ \mu $V;
\item Max out the number of samples and minimize number of records;
\end{itemize}

\section{Making a DLL}
\begin{framed}\noindent
  To make DLL suppose the project name is \texttt{Ilya\_Dll}
\end{framed}
\begin{enumerate}
  \begin{framed}\noindent
    \quote{\#ifdef  ILYA\_DLL\_EXPORTS}   \hfill  \quote{\red{derived
        from the project name}}

    \texttt{\#define        ANYNAME\_declspec(dllexport)}        \hfill
    \quote{\red{this macro is use when DLL is being made by VS}}

    \texttt{\#else}

    \texttt{\#define        ANYNAME\_declspec(dllimport)}        \hfill
    \quote{\red{this  macro is  used when  DLL is  used by  LabView to
        import the functions}}

    \texttt{extern "C" ANYNAME void name(parameters etc); }
  \end{framed}

\item Create .cpp  file, where the ADQAPI functions  are wrapped inside
  these exporting/importing functions;
\item Build \ira Build Ilya\_DLL;
\item \red{LabView  \ira \cmd{call  library function node}  \ira double
    click and navigate to the DLL, created in the projects debug folder
    WILL NOT WORK!!!!!}  !!!!!!

\begin{framed}\noindent
  \quote{ADQAPI.}   \hfill   \quote{\red{this  takes  the   DLL  from
      C:/Windows/SysWOW64/ADQAPI.dll}.   The  star  makes  this  import
    platform independent}

  Alternatively, I copied that DLL into  the LabView folder, gaves it a
  name  such as  Imperium\_ADQ.dll and  imported `Imperium.dll'  and it
  worked
\end{framed}
\item Or just use  the libraries made by SPDevices. I  am using the DLL
  in ADQAPI folder!!  For declaring the parameters types (replicate the
  names from the function data sheet)

\begin{framed}\noindent
  Thread: Run in UI thread

  Calling convention: C

  Error checking: default

  void*  \ira numeric,  unsigned 32-bit  integer, value  (adq\_cu\_ptr)
  uint32\_t

  unsigned  int \ira  numeric,  unsigned  32-bit integer  (adq214\_num)
  uint32\_t

  int \ira numeric, singed 32-bit integer int32\_t
\end{framed}
\end{enumerate}


\newpage

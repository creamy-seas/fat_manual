\section{RHUL Probe station (for measuring contact resistance)}
  \red{Turn on the microscope and the voltage source}
  \begin{itemize}
  	\item Place sample on bit of foam to prevent leakage to ground;
  	\item \red{Ground yourself};
  	\item \cmd{Move sample with red pin underneath};
  	\item \texttt{Turn small vertical handles} (on the inner side of the red handles) for positioning the pins roughly;
  	\item \cmd{Lower pins by lowering handles facing the wall} - bring into close proximity with surface;
  	\item Turn \texttt{red 10/4} revs for fine movement and to lift and lower pins. Look through the LHS objective of the microscope. \red{Do not overtighten!!! The only way you will know that the limit is reached is by the `springy' movement of sample - this is the queue to stop tightening};
  	\item \red{If red handles no longer turn, lift the pins up with the `wall facing' handles and recenter the dial by turning it to \cmd{no of max turns/2}};
  	\item \cmd{Launch LabView program}
  \end{itemize}
  \red{Once done, turn off the Keithley voltage source and turn off the white plug that powers the lamp.}
 
 \newpage
 